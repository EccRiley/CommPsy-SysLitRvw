\documentclass[11pt,]{tufte-handout}

% ams
\usepackage{amssymb,amsmath}

\usepackage{ifxetex,ifluatex}
\usepackage{fixltx2e} % provides \textsubscript
\ifnum 0\ifxetex 1\fi\ifluatex 1\fi=0 % if pdftex
  \usepackage[T1]{fontenc}
  \usepackage[utf8]{inputenc}
\else % if luatex or xelatex
  \makeatletter
  \@ifpackageloaded{fontspec}{}{\usepackage{fontspec}}
  \makeatother
  \defaultfontfeatures{Ligatures=TeX,Scale=MatchLowercase}
  \makeatletter
  \@ifpackageloaded{soul}{
     \renewcommand\allcapsspacing[1]{{\addfontfeature{LetterSpace=15}#1}}
     \renewcommand\smallcapsspacing[1]{{\addfontfeature{LetterSpace=10}#1}}
   }{}
  \makeatother
\fi

% graphix
\usepackage{graphicx}
\setkeys{Gin}{width=\linewidth,totalheight=\textheight,keepaspectratio}

% booktabs
\usepackage{booktabs}

% url
\usepackage{url}

% hyperref
\usepackage{hyperref}

% units.
\usepackage{units}


\setcounter{secnumdepth}{-1}

% citations

% pandoc syntax highlighting

% longtable

% multiplecol
\usepackage{multicol}

% strikeout
\usepackage[normalem]{ulem}

% morefloats
\usepackage{morefloats}


% tightlist macro required by pandoc >= 1.14
\providecommand{\tightlist}{%
  \setlength{\itemsep}{0pt}\setlength{\parskip}{0pt}}

% title / author / date
\title{Place-Based Methods - Reading Notes}
\author{Rachel M. Smith}
\date{Winter, 2016}

% \usepackage{caption}
% \usepackage{cleveref}
% \usepackage{biblatex}
% \renewbibmacro*{date}{%
%    \printdate
%    \iffieldundef{origyear}{%
%    }{%
%      \setunit*{\addspace}%
%      \printtext[parens]{\printorigdate}%
%    }%
% }

%
% --------------------- %
% Latex Logo Commands
% --------------------- %
%
\usepackage{xspace}
\newcommand{\latex}{\LaTeX\xspace}
\newcommand{\tex}{\TeX\xspace}
\newcommand{\bibtex}{\textsc{Bib}\tex}
%
% --------------------- %
% Colors
% --------------------- %
%
\usepackage{color}
\definecolor{magenta}{rgb}{0.79, 0.08, 0.48} %% ~c9147a %%
\definecolor{dkmagenta}{rgb}{0.55, 0.0, 0.55} %% ~8c008c %%
\definecolor{dpmagenta}{rgb}{0.8, 0.0, 0.8} %% ~140014 %%
\definecolor{patriarch}{rgb}{0.5, 0.0, 0.5} %% ~0d000d %%
\definecolor{dkpatriarch}{rgb}{0.4, 0.0, 0.4} %% ~0a000a %%
\definecolor{blue}{rgb}{0.07, 0.04, 0.56} %% ~120a8f %%
\definecolor{royalblue}{rgb}{0.0, 0.22, 0.66} %% ~0038a8 %%
\definecolor{dkblue}{rgb}{0.0, 0.0, 0.55} %% ~00008c %%
\definecolor{mnblue}{rgb}{0.1, 0.1, 0.44} %% ~030370 %%
\definecolor{smblue}{rgb}{0.0, 0.2, 0.6} %% ~00050f %%
\definecolor{Rblue}{rgb}{0.39, 0.35, 0.639} %% ~0a09a3 %%
\definecolor{dkmnblue}{rgb}{0.0, 0.2, 0.4} %% ~00050a %%
\definecolor{navy}{rgb}{0.0, 0.0, 0.5} %% ~00000d %%
\definecolor{dknavy}{rgb}{0, 0, .208} %% ~000035 %%
\definecolor{blublk}{rgb}{0, 0, .106} %% ~00001b %%
\definecolor{blugray}{rgb}{0.33, 0.41, 0.47} %% ~546978 %%
\definecolor{grayblue}{rgb}{0.33, 0.41, 0.58} %% ~~546994 %%
\definecolor{slgray}{rgb}{0.44, 0.5, 0.56} %% ~70808d %%
\definecolor{red}{rgb}{.545, 0.0, 0.0} %% ~8b0000 %%
\definecolor{dkred}{rgb}{.247, 0.0, 0.0} %% ~3f0000 %%
\definecolor{mplblu}{HTML}{363283}
%
\definecolor{pdxgray}{HTML}{373737} %% ~373737 %%
\definecolor{pdxgreen}{HTML}{8B9535} %% ~8B9535 %%
\definecolor{myblack}{HTML}{181C20} %% ~181C20 %%
%
% ---------------------------------------- %
% Indent first line of text in tabular env %
% ---------------------------------------- %
%
\newcommand{\rowgroup}[2][-1em]{\hspace{#1}#2}
\newcommand{\mrowgroup}[3]{\hspace*{#1}#2\hspace*{#1}#3}
%
% --------------------- %
% Format Block Quotes
% --------------------- %
%   Size: Scriptsize
%   Reduce vertical space above
%   Color: Gray
%
\usepackage{setspace}
% \expandafter\def\expandafter\quote\expandafter{\quote\small\singlespacing\color{myblack!65}\vspace{-0.5\baselineskip}}

% \expandafter\def\expandafter\quote\expandafter{\quote\small\singlespacing\vspace{-1em}}

% \setlength\listindent{1em}

\usepackage{enumitem}
% \setlist[itemize, 1]{leftmargin=!, labelindent=0.5em, itemindent=-3em, label=\scriptsize{$\cdot$}, partopsep=0em, topsep=0.15em}
% \setlist[itemize, 2]{leftmargin=4em, label=$\centerdot$, topsep=0em}
% \setlength{\itemindent}{5in}

%
% ------------------- %
% Make Links Standout
% ------------------- %
%   (E.Tufte does not believe in using colors in links. I disagree.) %
%
% \newcommand{\rurl}[1]{\underline{\color{dkblue}{\url{~1}}}}
% \newcommand{\rhref}[2]{\underline{\color{dkblue}{\href{~1}{~2}}}}
\hypersetup{breaklinks=true,colorlinks=true,linkcolor=navy,urlcolor=navy}
%
% ------------------- %
% Format "texttt"
% ------------------- %
%
\newcommand{\rtt}[1]{\color{patriarch}{\texttt{#1}}}
%
\usepackage{amsmath}

\usepackage{enumitem,amssymb}
\newlist{todolist}{itemize}{2}
\setlist[todolist]{label=$\square$}
\newcommand{\todoitem}[1]{\textit{\color{red}{#1}}}

\newcommand{\textbft}[1]{\underline{\textbf{\texttt{#1}}}}


% ---------------------------%
% Code Formatting %
% ---------------------------%
% \usepackage{highlight}

% \definecolor{fgcolor}{rgb}{0.196, 0.196, 0.196}
% \newcommand{\hlnum}[1]{\textcolor[rgb]{0.063,0.58,0.627}{#1}}%
% \newcommand{\hlstr}[1]{\textcolor[rgb]{0.063,0.58,0.627}{#1}}%
% \newcommand{\hlcom}[1]{\textcolor[rgb]{0.588,0.588,0.588}{#1}}%
% \newcommand{\hlopt}[1]{\textcolor[rgb]{0.196,0.196,0.196}{#1}}%
% \newcommand{\hlstd}[1]{\textcolor[rgb]{0.196,0.196,0.196}{#1}}%
% \newcommand{\hlkwa}[1]{\textcolor[rgb]{0.231,0.416,0.784}{#1}}%
% \newcommand{\hlkwb}[1]{\textcolor[rgb]{0.627,0,0.314}{#1}}%
% \newcommand{\hlkwc}[1]{\textcolor[rgb]{0,0.631,0.314}{#1}}%
% \newcommand{\hlkwd}[1]{\textcolor[rgb]{0.78,0.227,0.412}{#1}}%

\let\hlipl\hlkwb
% \newcommand{\Rrule}{\textcolor{Rblue}{\rule{\linewidth}{0.05mm}}\newline\includegraphics[width=0.5cm]{auxDocs/Rlogo.png}}
\usepackage{dashrule}

\newcommand{\Rrule}{
    % \setlength{\parindent}{-10pt}
    \vspace*{1em}
    \noindent
    \hspace{-1em}
    \includegraphics[width=0.5cm]{auxDocs/Rlogo.png}
    \textcolor{Rblue}{
        \rule[0.1in]{0.90\linewidth}{0.02mm}
    }
    \vspace{-1.35em}
}

\newcommand{\Rerule}{
    % \setlength{\parindent}{-0.5in}
    \noindent
    \hspace{-1em}
    \textcolor{Rblue}{
        $\llcorner$\rule[-0.4mm]{\linewidth}{0.02mm}
                % \hfill
                % $\lrcorner$
    }
}

\newcommand{\Rruleb}{
    % \setlength{\parindent}{-10pt}
    \vspace*{1em}
    \noindent
    \hspace{-1em}
    \includegraphics[width=0.5cm]{auxDocs/Rlogo-bw.png}
    \textcolor{slgray}{
        \rule[0.1in]{0.90\textwidth}{0.02mm}
    }
    \vspace{-1.35em}
}

\newcommand{\Reruleb}{
    % \setlength{\parindent}{-0.5in}
    \noindent
    \hspace{-1em}
    \textcolor{slgray}{
        $\llcorner$\rule[-0.4mm]{\textwidth}{0.02mm}
                % \hfill
                % $\lrcorner$
    }
}

\newcommand{\MPrule}{
    % \setlength{\parindent}{-10pt}
    \vspace*{1em}
    \noindent
    \hspace{-1em}
    \includegraphics[width=0.5cm]{../auxDocs/mplus.png}
    \textcolor{mplblu}{
        \rule[0.1in]{0.90\textwidth}{0.02mm}
    }
    \vspace{-1.5em}
}

\newcommand{\MPerule}{
    % \setlength{\parindent}{-0.5in}
    \noindent
    \hspace{-1em}
    \textcolor{mplblu}{
        $\llcorner$\rule[-0.4mm]{\textwidth}{0.02mm}
                % \hfill
                % $\lrcorner$
    }
}

\newcommand{\Frule}{
    \vspace*{-1em}
    \begin{fullwidth}\textcolor{blublk}{\rule{\linewidth}{0.2mm}}\end{fullwidth}
}

\newcommand{\Frnp}{
    \vspace*{-1em}
    \begin{fullwidth}\textcolor{blublk}{\rule{\linewidth}{0.2mm}}\end{fullwidth}
    \newpage
}

\newcommand{\Fbrule}{ %% for tufte book format %%
    \vspace*{-4em}
    \begin{fullwidth}\textcolor{blublk}{\rule{\linewidth}{0.2mm}}\end{fullwidth}
}

% \newcommand{\Rrule}{
%     \noindent
%     \textcolor{Rblue}{
%         $\ulcorner\textregistered$\hdashrule[0.015in]{
%             0.9\linewidth
%             }
%             {1pt}{1pt}
%         $\textregistered\urcorner$
%         }}
% \newcommand{\Rerule}{
%     \noindent
%     \textcolor{Rblue}{
%         $\llcorner\textregistered$\hdashrule[0.025in]{
%             0.9\linewidth
%             }
%             {1pt}{1pt}
%         $\textregistered\lrcorner$
%         }}
% \newcommand{\Rerule}{\noindent\textcolor{Rblue}{\vdash\hdashrule[-0.015in]{\linewidth}{1pt}{1pt}\dashv}

% \vdash - \dashv
% \perp
% \ll - \gg
% +
% \pm
% \mp
% \ \newcommand{\Rrule}{\noindent\includegraphics[width=0.5cm]{auxDocs/Rlogo.png}\textcolor{Rblue}{\hdashrule[0.25in]{\linewidth}{1pt}{1pt}}}
% \newcommand{\Rrule}{\noindent\includegraphics[width=0.5cm]{auxDocs/Rlogo.png}\textcolor{Rblue}{\rule[0.25in]{\linewidth}{0.05mm}}}

% \newcommand{\Rerule}{\textcolor{Rblue}{\rule{\linewidth}{0.05mm}}}
% \hdashrule [⟨raise⟩] [⟨leader⟩] {⟨width⟩} {⟨height⟩} {⟨dash⟩}

%%%%%%%%%%%%%%%%%%%%%%%%%%
% Wrapper for Chi-Square %
%%%%%%%%%%%%%%%%%%%%%%%%%%

\newcommand{\tdef}[3][-0.5em]{\tufteskip\noindent\rowgroup[#1]{\textsc{#2}} \newline {#3}}
\newcommand{\hf}{\hfill}
\DeclareMathAlphabet{\mathpzc}{OT1}{pzc}{m}{it} %% to make \mathpzc typeset its argument in Zapf Chancery (see page 16 of "The Great, Big List of LATEX Symbols" by David Carlisle, Scott Pakin, & Alexander Holt (2001)) %%

\newcommand{\chisq}{\mathpzc{\chi^{2}}}

\newcommand{\sq}{^{2}}

\newcommand{\df}{\mathpzc{df}} %% degrees of freedom (df) %%

%
% ----------------------------- %
% Command to insert "ToDo" tags
% ----------------------------- %

\newcommand{\todo}{
    \textsuperscript{
        \tiny{
            \fcolorbox{dkred}{black!10}{
                \color{red}{
                    \textbf{\texttt{[ToDo]}}
                }
            }
        }
    }
}

\newcommand{\inprogress}{
    \textcolor{blue}{
        \textbf{\textit{\texttt{[In Progress]}}}
    }
}

\newcommand{\complete}{
    \sout{
        \textcolor{slgray}{
            \textit{\texttt{[Complete]}}
        }
    }
}

\newcommand{\edit}[1]{
    \textcolor{red}{
        \texttt{#1}
    }
}

\newcommand{\todot}{
    \textcolor{red}{\Large{$\mathbf{^{\otimes}}$}}
}

% \usepackage{wrapfig}

\newcommand{\refs}{
    \parindent=-1.7em\
    \
    \setlength{\parskip}{0.5\baselineskip}
}

%%%%%%%%%%%%%%%%%%%%%%%%%%%%%%%%%%%%%%%%%%%%%%%%%%%%%%%%%%%%%%%%%%%%%%%%%%%%%%%%%%%%
% the lines below are based on this SO answer: https://tex.stackexchange.com/a/48212
\makeatletter
\newcommand\semiHuge{\@setfontsize\semiHuge{20.82}{24.98}}
\makeatother

\makeatletter
\newcommand\LArge{\@setfontsize\LArge{17.35185}{20.82}}
\makeatother
%%%%%%%%%%%%%%%%%%%%%%%%%%%%%%%%%%%%%%%%%%%%%%%%%%%%%%%%%%%%%%%%%%%%%%%%%%%%%%%%%%%%

\begin{document}

\maketitle



{
\setcounter{tocdepth}{1}
\tableofcontents
}

\section{Proshansky, Fabian, \& Kaminoff
(1983)}\label{proshansky1983place}

In their discussion of the \emph{Anxiety and Defense Functions} of
place-identity, Proshansky et al. (1983) assert that ``\ldots{}
persistent cognitive discrepancies between physical setting properties
and place-identity expectancies may foster feelings of threat and pain
in association with those cognitions which in turn motivate the person
to avoid this setting. Reactions of this kind will often occur in the
formative years but are clearly not limited to childhood.'' (p.~73) I am
curious about the effect of such discrepancies occurring at a slightly
more distinct level with two particular components: (1) prolonged (i.e.,
far beyond the formative developmental years) experiences of cognitive
discrepancies related to one's place-identity, and (2) discrepancies
that occur in places or spaces, or in close association with either, to
which an individual feels she is either supposed to belong based on one
or multiple facets of her identity. In less abstract terms, I am
thinking specifically about my research on lesbian, bisexual,
transgender, and queer (LBTQ) women and girls, and experiences of
marginalization \emph{within} LBTQ-specific spaces and places. While my
focus until about six months ago has been intersectionally-based, I've
recently realized through conducting interviews with other LBTQ women
that my intersectional focus has been limited to the individual
personality and demographic (for lack of a better word at the moment)
facets of identity without taking into account the intersections that
exist among places, spaces, and individual identities.

\section{Lewicka (2011)}\label{lewicka2011place}

Interestingly, the Lewicka (2011) discusses findings indicating that
research has found the presence of diversity to be negatively related to
place attachment, community cohesion, interpersonal trust, etc. However,
I wonder if the same is true for a lack of diversity. Thinking of my own
experiences and Portland's overwhelming lack of racial diversity, and
the continued issues of gentrification across the city, I have often
found myself lacking in a sense of community tied to Portland and part
of that has been because I am a bit hyperaware of the fact that I am
more often than not interacting with people that look a lot like me (in
terms of race, socio-economic status, and age). My hyperawareness of
this dynamic is, in my own analysis, in large part due to the fact that
people that look like me tend to carry with them a lot of privilege,
especially in Portland, OR. I wonder, then, if there is another factor
involved in the relationship(s) between diversity and place attachment.
In particular, I wonder what historical socio-political factors might be
useful in this particular realm of analysis in place research.

\section{McMillan \& Chavis (1986)}\label{mcmillan1986sense}

In the thirty years since this article was published, we have witnessed
several manifestations, both here in the U.S. and globally, of the
``potential for great social conflict'' (McMillan \& Chavis, 1986, p.
20) resulting in part from increased senses of community among
relatively small, but loud, organized communities. We have also seen a
great deal of technological advancements, particularly in terms of the
people's ability to create, foster, and join communities online as well
as in person. What implications do these historical dynamics have for
community psychologists in terms of understanding sense of community as
it manifests in our world today, as well as the application of that
understanding?

\section{Pretty, Chipuer, \& Bramston (2003)}\label{pretty2003sense}

In their examination of discriminate factors associated with dimensions
of sense of community Pretty et al. (2003) found age as a potential key
factor. The authors assert that ``This implies that there are other
aspects of community sentiment not included here which are important to
an adolescent's feeling that `this town is the place for me'.
Alternatively this finding may indicate that other objective dimensions
of a community, such as its economic opportunities, may be more
instrumental to identity during the adolescent life stage'' (p.~283). Is
there also a possibility that the social and physical environments in
either or both of the study's settings could have underwent or were
currently undergoing changes in terms of the people joining or leaving
the communities? How might the introduction or loss of community members
in these relatively small and rural places also impact adolescents',
versus adults', senses of community?

\section{Stokols (1992)}\label{stokols1992establishing}

I appreciated Stokols (1992) discussion of public policy implementation
and I am curious how can we write and effectively implement legislation
that is more primary prevention focused in its emphasis on the actual
root causes of whatever is being prevented?

\section{Kloos \& Shah (2009)}\label{kloos2009social}

I appreciated Kloos \& Shah (2009) emphasis on the importance of the
interaction between individuals and their immediate home/housing
environment. In thinking about housing for individuals with serious
mental illness that is integrated within the surrounding neighborhood or
community, I was reminded of a soon-to-be opened school specifically for
lesbian, gay, bisexual, transgender, and queer (LGBTQ) youth in Atlanta,
GA. I've struggled a bit in deciding on my opinion about this particular
school because it seems that the intended benefits of such a school
(i.e.~Safety for LGBTQ youth) is being sought at the expense of
community integration. What are the potential negative consequences of
disintegrating marginalized populations from their surrounding social
environments?

\section{Wright \& Kloos (2007)}\label{wright2007housing}

Wright \& Kloos (2007) argue, and in my opinion provide sufficient
evidence through their findings, that individuals' perceptions of their
immediate physical environment impacts individuals their experiences
within those environments. I am curious, however, about the intensity,
rather than immediacy, of individuals perceptions of their environments.
What is the potential effect of differentially intense perceptions of
more distal aspects of an individual's setting?

\section{Shinn \& Toohey (2003)}\label{shinn2003community}

Shinn \& Toohey (2003) assert that ``Diverse groups may feel
differentially accepted and supported in the same setting'' (p.~443). In
the examples that follow this assertion, the authors cite research
findings regarding LGBTQ individuals' experiences of marginalization in
certain settings despite heterosexual individuals' perceiving the same
settings as open to and accepting of LGBTQ individuals. I am once again
reminded of within group marginalization among marginalized populations.
Is it possible that settings that being in setting that is experienced
as marginalizing by a whole group overshadows the potential experiences
of marginalization within that group?

\section{Nicotera (2007)}\label{nicotera2007measuring}

Nicotera (2007) provides a sort of spectrum of conceptualizations of the
neighborhood construct ranging from more rigid definitions toward more
flexible conceptualizations. The fourth definition listed in Table 1
(p.~30), in my view, represents the latter end of this spectrum. As I
continued through the paper, I was left with these questions: (1) What
are the potential benefits and drawbacks of having a more flexible
definition of neighborhood? (2) How does a flexible versus more rigid
definition (and vice-versa) of neighborhood benefit applied
psychological research?

\section{Sullivan (2006)}\label{sullivan2006assessing}

In discussing his study's findings Sullivan (2006) argues that the study
contributes to the current body of relevant research literature by
``accurately measuring the level of neighborhood support for
gentrification using survey data, probability sampling, and a large
sample'' (p.~617). I was genuinely surprised to see this the first of
two contributions the study's findings makes to research on
neighborhoods and gentrification given that the measurement of
residents' attitudes toward ongoing gentrification processes in the
Alberta Street neighborhood was precisely where I felt the author did
not provide sufficient detail in explaining the study's methods. Based
on the author's descriptions of the measures used for this study, which
provides very little in the way of how the interview questions were
derived and the reliability and validity of the items used to measure
each of the study's variables, I would argue that resident's attitudes
were somewhat, but not entirely, adequately measured based on the set
standards of reliability and representativeness in terms of
quantitatively measuring a thing, but that this is not necessarily an
accurate measure of residents' attitudes given that the author does not
sufficiently defend the validity of measures' content.

\section{Sullivan \& Shaw (2011)}\label{sullivan2011retail}

One of the longtime critiques of gentrification is that it is a process
that tends to displace already marginalized groups within a given
setting. That ongoing marginalization process works primarily by
silencing the voices of the longtime residents of the area being
gentrified in favor of elevating the voices of the gentrifiers, who, as
discussed by both Sullivan (2006) and Sullivan \& Shaw (2011), tend to
be representative of more affluent/dominant social and economic classes.
I find it interesting, then, that both Sullivan (2006) and Sullivan \&
Shaw (2011) focus primarily, if not exclusively, on black and white
residents in the Alberta Street neighborhood. Who's voices are silenced
and what identities are ignored in their analyses as a consequence of
their relatively limited sampling frame?

\section{Thomas, Pate, \& Ranson (2015)}\label{thomas2015crosstown}

I appreciated Thomas et al. (2015)`s discussion of their findings around
creative tensions: ``creative tensions emerged around different notions
of community and different ways of understanding art. \ldots{} These two
visions of community illustrate the tensions in place-making as a
project that occurs across multiple levels and is framed in contested
ways'' (p.~80). This case study, and the authors' analysis of the
effects on the immediately surrounding community of creating an arts
community center, reminds me of a ``grass roots'' community arts and
restoration movement that began a few years ago in the Edgewood, Old
Fourth Ward, and Reynoldstown neighborhoods of Atlanta, GA.
Collectively, these three particular neighborhoods happen represent one
of the epicenters of the U.S. Civil Rights Movement during the mid-1900s
(Dr.~Martin Luther Kind Jr.'s childhood home and Ebenezer Baptist Church
are both located near where the Edgewood and Old Fourth Ward
neighborhoods converge). When this community arts and restoration
movement, called ``Living Walls'', began I was a young college student
with a hunger for social justice activism attending Georgia State
University (which is located adjacent to the Edgewood Neighborhood), had
lived in various spots within the Edgewood and Old Fourth Ward
neighborhoods, and spending more than a fair amount of time at my
favorite coffee shop located in Reynoldstown. I was also a
stereotypically white social activist with a still relatively limited
understanding of intersectionality. So, I thought the movement was
``super cool!''. It was not until my partner at the time called the
movement ``exceedingly pretentious'' that I realized what was actually
going on with this movement: a bunch of people with both social and
economic influence were promoting their view of what was positive
artistic expression at the expense of damning the use of ``typical''
graffiti in these neighborhoods. As time went on, I chose not to
participate directly in the movement, and instead watched and listened
to my friends who ended up standing on either side of it (i.e., those
friends who actively participated in the movement and those who actively
resisted it). What I ended up realizing was that the ``typical''
graffiti that the movement's leaders had deemed to be a negative aspect
of the community was in fact the exact opposite to the community's
longterm inhabitants.

Thomas et al. (2015)`s observations around creative tensions, and the
effects of those tensions on place-making as well as on community
members' differential senses of community, resonated quite saliently
with my own, non-research-based experiences of observing similar
community building and restoration efforts in Atlanta, GA. For me, my
own experiences coupled with my experience reading Thomas et al.
(2015)'s case study highlight the importance of anticipating,
understanding, and appropriately reacting to the potential negative
consequences of social activism and action research endeavors.

\section{Wang \& Burris (1997)}\label{wang1997photovoice}

Wang \& Burris (1997) quite appropriately, in my opinion, argue that
``The participatory process attempts to address material and status
inequalities, yet the extent to which it may perpetuate those
inequalities deserves scrutiny'' (p.~374). I agree. However, the authors
later discuss the capacity building element of photovoice (see p.~375).
First, I was intrigued by this particular element, because I had never
really considered photovoice as a method involving any sort of capacity
building. Based on Wang \& Burris (1997) description, photovoice as part
of needs assessment efforts makes sense as a potentially capacity
building tool for community action; however, I could not help but remain
skeptical, given that the participants in the authors' example were
given tools that facilitated the formation of the Provincial and County
Guidance Group in Yunnan, and those tools were provided by researchers.
That's really great and all, but I worry about what happens when the
researchers, and the various power and privileges they represent and
bring with themselves, leave or move on to other interests. To what
extent does providing a set of tools to the members of a community
remove or address issues of power and privilege, particularly power and
privilege among those who provided the tools (i.e., the researchers)?

\section{Smith, Padgett, Choy-Brown, \& Henwood
(2015)}\label{smith2015rebuilding}

In their overview description of the themes found among participants'
photographs Smith et al. (2015) note that female participants, who
represent less than 20\% of the study's sample, took the majority of
photos of people, and that all of the people in those photos were those
with whom the participants had close interpersonal relationships. I
found this particularly interesting given that the sub-sample size of
female participants was rather small (which I understand is likely a
function, at least in part, of the distribution of males and females in
the study's target population), and yet these participants' photographs
represented a majority theme distinguishable from themes represented in
the men's photographs. I wonder if there is something here (for future
directions perhaps) in terms of the potentially differential importance
of social/interpersonal relationships among men and women. Social
psychological research, as well research from clinical and counseling
psychology, has found significant evidence that men are socialized to
adhere to specific masculine gender roles, one of which is extreme
self-reliance (Levant \& Richmond, 2007; e.g., Levant, Rankin, Williams,
Hasan, \& Smalley, 2010).

In light of Smith et al. (2015) other findings related to the emphasis
on economic well-being among participants (e.g., having one's own
apartment, future job possibilities/hopes, etc.), I think there might be
room for future research to examine how the construct of place
intersects with individuals' gender role socialization and gender role
adherence.

\section{Case, Todd, \& Kral (2014)}\label{case2014ethnography}

There is this saying among academics that ``research is me-search'',
usually referring to the idea that the phenomena researchers choose to
examine most closely are those with which they have some close personal
experience or investment. I was reminded of this saying when reading
Case et al. (2014) description of the tension in ethnographic research
within community psychology around ``Optimal Levels of Engagement''
(p.~65). I was not, however, thinking of researchers' personal
experiences with or investments in their chosen phenomena of interest,
but really the fact that there is very difficult to actually remove the
researcher from the research. In thinking further on this, I was also
reminded of Grounded Theory research and analytic methods, and the
emphasis within this method on the investigator's interpretation of the
data as one of the main building blocks in theory development. I wonder,
then, the issue of optimal levels of engagement away might be in part
resolved/addressed by researcher's intentionally explaining their
experience engaging a given context in their research and the potential
meaning of that engagement in terms of their findings.

\section{Jason \& Glenwick (2016)}\label{dutta2016ethnographic}

Jason \& Glenwick (2016) provides a really nice explanation and
illustration of how research cannot be removed or realistically
considered objectively separate from the researchers of any given
phenomenon. While the purpose of Jason \& Glenwick (2016) chapter seems
primarily illustrative, thus the emphasis on the researcher's impact on
his research and vise-versa, I wonder how researchers conducting
non-ethnographic inquiries could incorporate information or reflections
in the write-up of a given study about their role and impact on their
researched phenomena, research participants, and the context(s)
surrounding their research.

\newpage

\section*{References}\label{references}
\addcontentsline{toc}{section}{References}

\hypertarget{refs}{}
\hypertarget{ref-case2014ethnography}{}
Case, A. D., Todd, N. R., \& Kral, M. J. (2014). Ethnography in
community psychology: Promises and tensions. \emph{American Journal of
Community Psychology}, \emph{54}, 60--71.

\hypertarget{ref-dutta2016ethnographic}{}
Jason, L. A., \& Glenwick, D. S. (Eds.). (2016). Ethnographic
approaches. In \emph{Handbook of methodological approaches to
community-based research: Qualitative, quantitative, and mixed methods}.
New York, NY: Oxford University Press.

\hypertarget{ref-kloos2009social}{}
Kloos, B., \& Shah, S. (2009). A social ecological approach to
investigating relationships between housing and adaptive functioning for
persons with serious mental illness. \emph{American Journal of Community
Psychology}, \emph{44}, 316--326.

\hypertarget{ref-levant2007review}{}
Levant, R. F., \& Richmond, K. (2007). A review of research on
masculinity ideologies using the Male Role Norms Inventory. \emph{The
Journal of Men's Studies}, \emph{15}, 130--146.

\hypertarget{ref-levant2010evaluation}{}
Levant, R. F., Rankin, T., Williams, C., Hasan, N., \& Smalley, B.
(2010). Evaluation of the factor structure and construct validity of
scores on the Male Role Norms Inventory-Revised (MRNI-r).
\emph{Psychology of Men and Masculinity}, \emph{11}, 25--37.

\hypertarget{ref-lewicka2011place}{}
Lewicka, M. (2011). Location attachment: How far have we come in the
last 40 years? \emph{Journal of Environmental Psychology}, \emph{31},
207--230.

\hypertarget{ref-mcmillan1986sense}{}
McMillan, D. W., \& Chavis, D. M. (1986). Sense of community: A
definition and theory. \emph{Journal of Community Psychology},
\emph{14}, 6--23.

\hypertarget{ref-nicotera2007measuring}{}
Nicotera, N. (2007). Measuring neighborhood: A conundrum for human
services researchers and practitioners. \emph{American Journal of
Community Psychology}, \emph{40}, 26--51.

\hypertarget{ref-pretty2003sense}{}
Pretty, G. H., Chipuer, H. M., \& Bramston, P. (2003). Sense of place
amongst adolescents and adults in two rural australian towns: The
discriminating features of place attachment, sense of community and
place dependence in relation to place identity. \emph{Journal of
Environmental Psychology}, \emph{23}, 273--287.

\hypertarget{ref-proshansky1983place}{}
Proshansky, H. M., Fabian, A. K., \& Kaminoff, R. (1983).
Location-identity: Physical world socialization of the self.
\emph{Journal of Environmental Psychology}, \emph{3}, 57--83.

\hypertarget{ref-shinn2003community}{}
Shinn, M., \& Toohey, S. M. (2003). Community contexts of human welfare.
\emph{Annual Review of Psychology}, \emph{54}, 427--459.

\hypertarget{ref-smith2015rebuilding}{}
Smith, B. T., Padgett, D. K., Choy-Brown, M., \& Henwood, B. F. (2015).
Rebuilding lives and identities: The role of place in recovery among
persons with complex needs. \emph{Health \& Place}, \emph{33}, 109--117.

\hypertarget{ref-stokols1992establishing}{}
Stokols, D. (1992). Establishing and maintaining healthy environments:
Toward a social ecology of health promotion. \emph{American
Psychologist}, \emph{47}, 6.

\hypertarget{ref-sullivan2006assessing}{}
Sullivan, D. M. (2006). Assessing residents opinions on changes in a
gentrifying neighborhood: A case study of the alberta neighborhood in
portland, oregon. \emph{Housing Policy Debate}, \emph{17}, 595--624.

\hypertarget{ref-sullivan2011retail}{}
Sullivan, D. M., \& Shaw, S. C. (2011). Retail gentrification and race:
The case of alberta street in portland, oregon. \emph{Urban Affairs
Review}, \emph{47}, 413--432.

\hypertarget{ref-thomas2015crosstown}{}
Thomas, E., Pate, S., \& Ranson, A. (2015). The crosstown initiative:
Art, community, and locationmaking in memphis. \emph{American Journal of
Community Psychology}, \emph{55}, 74--88.

\hypertarget{ref-wang1997photovoice}{}
Wang, C., \& Burris, M. A. (1997). Photovoice: Concept, methodology, and
use for participatory needs assessment. \emph{Health Education \&
Behavior}, \emph{24}, 369--387.

\hypertarget{ref-wright2007housing}{}
Wright, P. A., \& Kloos, B. (2007). Housing environment and mental
health outcomes: A levels of analysis perspective. \emph{Journal of
Environmental Psychology}, \emph{27}, 79--89.



\end{document}
