\documentclass[]{tufte-handout}

% ams
\usepackage{amssymb,amsmath}

\usepackage{ifxetex,ifluatex}
\usepackage{fixltx2e} % provides \textsubscript
\ifnum 0\ifxetex 1\fi\ifluatex 1\fi=0 % if pdftex
  \usepackage[T1]{fontenc}
  \usepackage[utf8]{inputenc}
\else % if luatex or xelatex
  \makeatletter
  \@ifpackageloaded{fontspec}{}{\usepackage{fontspec}}
  \makeatother
  \defaultfontfeatures{Ligatures=TeX,Scale=MatchLowercase}
  \makeatletter
  \@ifpackageloaded{soul}{
     \renewcommand\allcapsspacing[1]{{\addfontfeature{LetterSpace=15}#1}}
     \renewcommand\smallcapsspacing[1]{{\addfontfeature{LetterSpace=10}#1}}
   }{}
  \makeatother
\fi

% graphix
\usepackage{graphicx}
\setkeys{Gin}{width=\linewidth,totalheight=\textheight,keepaspectratio}

% booktabs
\usepackage{booktabs}

% url
\usepackage{url}

% hyperref
\usepackage{hyperref}

% units.
\usepackage{units}


\setcounter{secnumdepth}{-1}

% citations

% pandoc syntax highlighting

% longtable

% multiplecol
\usepackage{multicol}

% strikeout
\usepackage[normalem]{ulem}

% morefloats
\usepackage{morefloats}


% tightlist macro required by pandoc >= 1.14
\providecommand{\tightlist}{%
  \setlength{\itemsep}{0pt}\setlength{\parskip}{0pt}}

% title / author / date
\title{MAP Annotated Bibliography - NEW}
\date{06 April 2017}

% \usepackage{caption}
% \usepackage{cleveref}
% \usepackage{biblatex}
% \renewbibmacro*{date}{%
%    \printdate
%    \iffieldundef{origyear}{%
%    }{%
%      \setunit*{\addspace}%
%      \printtext[parens]{\printorigdate}%
%    }%
% }

%
% --------------------- %
% Latex Logo Commands
% --------------------- %
%
\usepackage{xspace}
\newcommand{\latex}{\LaTeX\xspace}
\newcommand{\tex}{\TeX\xspace}
\newcommand{\bibtex}{\textsc{Bib}\tex}
%
% --------------------- %
% Colors
% --------------------- %
%
\usepackage{color}
\definecolor{magenta}{rgb}{0.79, 0.08, 0.48} %% ~c9147a %%
\definecolor{dkmagenta}{rgb}{0.55, 0.0, 0.55} %% ~8c008c %%
\definecolor{dpmagenta}{rgb}{0.8, 0.0, 0.8} %% ~140014 %%
\definecolor{patriarch}{rgb}{0.5, 0.0, 0.5} %% ~0d000d %%
\definecolor{dkpatriarch}{rgb}{0.4, 0.0, 0.4} %% ~0a000a %%
\definecolor{blue}{rgb}{0.07, 0.04, 0.56} %% ~120a8f %%
\definecolor{royalblue}{rgb}{0.0, 0.22, 0.66} %% ~0038a8 %%
\definecolor{dkblue}{rgb}{0.0, 0.0, 0.55} %% ~00008c %%
\definecolor{mnblue}{rgb}{0.1, 0.1, 0.44} %% ~030370 %%
\definecolor{smblue}{rgb}{0.0, 0.2, 0.6} %% ~00050f %%
\definecolor{Rblue}{rgb}{0.39, 0.35, 0.639} %% ~0a09a3 %%
\definecolor{dkmnblue}{rgb}{0.0, 0.2, 0.4} %% ~00050a %%
\definecolor{navy}{rgb}{0.0, 0.0, 0.5} %% ~00000d %%
\definecolor{dknavy}{rgb}{0, 0, .208} %% ~000035 %%
\definecolor{blublk}{rgb}{0, 0, .106} %% ~00001b %%
\definecolor{blugray}{rgb}{0.33, 0.41, 0.47} %% ~546978 %%
\definecolor{grayblue}{rgb}{0.33, 0.41, 0.58} %% ~~546994 %%
\definecolor{slgray}{rgb}{0.44, 0.5, 0.56} %% ~70808d %%
\definecolor{red}{rgb}{.545, 0.0, 0.0} %% ~8b0000 %%
\definecolor{dkred}{rgb}{.247, 0.0, 0.0} %% ~3f0000 %%
\definecolor{mplblu}{HTML}{363283}
%
\definecolor{pdxgray}{HTML}{373737} %% ~373737 %%
\definecolor{pdxgreen}{HTML}{8B9535} %% ~8B9535 %%
\definecolor{myblack}{HTML}{181C20} %% ~181C20 %%
%
% ---------------------------------------- %
% Indent first line of text in tabular env %
% ---------------------------------------- %
%
\newcommand{\rowgroup}[2][-1em]{\hspace{#1}#2}
\newcommand{\mrowgroup}[3]{\hspace*{#1}#2\hspace*{#1}#3}
%
% --------------------- %
% Format Block Quotes
% --------------------- %
%   Size: Scriptsize
%   Reduce vertical space above
%   Color: Gray
%
\usepackage{setspace}
% \expandafter\def\expandafter\quote\expandafter{\quote\small\singlespacing\color{myblack!65}\vspace{-0.5\baselineskip}}

% \expandafter\def\expandafter\quote\expandafter{\quote\small\singlespacing\vspace{-1em}}

% \setlength\listindent{1em}

\usepackage{enumitem}
% \setlist[itemize, 1]{leftmargin=!, labelindent=0.5em, itemindent=-3em, label=\scriptsize{$\cdot$}, partopsep=0em, topsep=0.15em}
% \setlist[itemize, 2]{leftmargin=4em, label=$\centerdot$, topsep=0em}
% \setlength{\itemindent}{5in}

%
% ------------------- %
% Make Links Standout
% ------------------- %
%   (E.Tufte does not believe in using colors in links. I disagree.) %
%
% \newcommand{\rurl}[1]{\underline{\color{dkblue}{\url{~1}}}}
% \newcommand{\rhref}[2]{\underline{\color{dkblue}{\href{~1}{~2}}}}
\hypersetup{breaklinks=true,colorlinks=true,linkcolor=navy,urlcolor=navy}
%
% ------------------- %
% Format "texttt"
% ------------------- %
%
\newcommand{\rtt}[1]{\color{patriarch}{\texttt{#1}}}
%
\usepackage{amsmath}

\usepackage{enumitem,amssymb}
\newlist{todolist}{itemize}{2}
\setlist[todolist]{label=$\square$}
\newcommand{\todoitem}[1]{\textit{\color{red}{#1}}}

\newcommand{\textbft}[1]{\underline{\textbf{\texttt{#1}}}}


% ---------------------------%
% Code Formatting %
% ---------------------------%
% \usepackage{highlight}

% \definecolor{fgcolor}{rgb}{0.196, 0.196, 0.196}
% \newcommand{\hlnum}[1]{\textcolor[rgb]{0.063,0.58,0.627}{#1}}%
% \newcommand{\hlstr}[1]{\textcolor[rgb]{0.063,0.58,0.627}{#1}}%
% \newcommand{\hlcom}[1]{\textcolor[rgb]{0.588,0.588,0.588}{#1}}%
% \newcommand{\hlopt}[1]{\textcolor[rgb]{0.196,0.196,0.196}{#1}}%
% \newcommand{\hlstd}[1]{\textcolor[rgb]{0.196,0.196,0.196}{#1}}%
% \newcommand{\hlkwa}[1]{\textcolor[rgb]{0.231,0.416,0.784}{#1}}%
% \newcommand{\hlkwb}[1]{\textcolor[rgb]{0.627,0,0.314}{#1}}%
% \newcommand{\hlkwc}[1]{\textcolor[rgb]{0,0.631,0.314}{#1}}%
% \newcommand{\hlkwd}[1]{\textcolor[rgb]{0.78,0.227,0.412}{#1}}%

\let\hlipl\hlkwb
% \newcommand{\Rrule}{\textcolor{Rblue}{\rule{\linewidth}{0.05mm}}\newline\includegraphics[width=0.5cm]{auxDocs/Rlogo.png}}
\usepackage{dashrule}

\newcommand{\Rrule}{
    % \setlength{\parindent}{-10pt}
    \vspace*{1em}
    \noindent
    \hspace{-1em}
    \includegraphics[width=0.5cm]{auxDocs/Rlogo.png}
    \textcolor{Rblue}{
        \rule[0.1in]{0.90\linewidth}{0.02mm}
    }
    \vspace{-1.35em}
}

\newcommand{\Rerule}{
    % \setlength{\parindent}{-0.5in}
    \noindent
    \hspace{-1em}
    \textcolor{Rblue}{
        $\llcorner$\rule[-0.4mm]{\linewidth}{0.02mm}
                % \hfill
                % $\lrcorner$
    }
}

\newcommand{\Rruleb}{
    % \setlength{\parindent}{-10pt}
    \vspace*{1em}
    \noindent
    \hspace{-1em}
    \includegraphics[width=0.5cm]{auxDocs/Rlogo-bw.png}
    \textcolor{slgray}{
        \rule[0.1in]{0.90\textwidth}{0.02mm}
    }
    \vspace{-1.35em}
}

\newcommand{\Reruleb}{
    % \setlength{\parindent}{-0.5in}
    \noindent
    \hspace{-1em}
    \textcolor{slgray}{
        $\llcorner$\rule[-0.4mm]{\textwidth}{0.02mm}
                % \hfill
                % $\lrcorner$
    }
}

\newcommand{\MPrule}{
    % \setlength{\parindent}{-10pt}
    \vspace*{1em}
    \noindent
    \hspace{-1em}
    \includegraphics[width=0.5cm]{../auxDocs/mplus.png}
    \textcolor{mplblu}{
        \rule[0.1in]{0.90\textwidth}{0.02mm}
    }
    \vspace{-1.5em}
}

\newcommand{\MPerule}{
    % \setlength{\parindent}{-0.5in}
    \noindent
    \hspace{-1em}
    \textcolor{mplblu}{
        $\llcorner$\rule[-0.4mm]{\textwidth}{0.02mm}
                % \hfill
                % $\lrcorner$
    }
}

\newcommand{\Frule}{
    \vspace*{-1em}
    \begin{fullwidth}\textcolor{blublk}{\rule{\linewidth}{0.2mm}}\end{fullwidth}
}

% \newcommand{\Rrule}{
%     \noindent
%     \textcolor{Rblue}{
%         $\ulcorner\textregistered$\hdashrule[0.015in]{
%             0.9\linewidth
%             }
%             {1pt}{1pt}
%         $\textregistered\urcorner$
%         }}
% \newcommand{\Rerule}{
%     \noindent
%     \textcolor{Rblue}{
%         $\llcorner\textregistered$\hdashrule[0.025in]{
%             0.9\linewidth
%             }
%             {1pt}{1pt}
%         $\textregistered\lrcorner$
%         }}
% \newcommand{\Rerule}{\noindent\textcolor{Rblue}{\vdash\hdashrule[-0.015in]{\linewidth}{1pt}{1pt}\dashv}

% \vdash - \dashv
% \perp
% \ll - \gg
% +
% \pm
% \mp
% \ \newcommand{\Rrule}{\noindent\includegraphics[width=0.5cm]{auxDocs/Rlogo.png}\textcolor{Rblue}{\hdashrule[0.25in]{\linewidth}{1pt}{1pt}}}
% \newcommand{\Rrule}{\noindent\includegraphics[width=0.5cm]{auxDocs/Rlogo.png}\textcolor{Rblue}{\rule[0.25in]{\linewidth}{0.05mm}}}

% \newcommand{\Rerule}{\textcolor{Rblue}{\rule{\linewidth}{0.05mm}}}
% \hdashrule [⟨raise⟩] [⟨leader⟩] {⟨width⟩} {⟨height⟩} {⟨dash⟩}

%%%%%%%%%%%%%%%%%%%%%%%%%%
% Wrapper for Chi-Square %
%%%%%%%%%%%%%%%%%%%%%%%%%%

\newcommand{\tdef}[3][-0.5em]{\tufteskip\noindent\rowgroup[#1]{\textsc{#2}} \newline {#3}}
\newcommand{\hf}{\hfill}
\DeclareMathAlphabet{\mathpzc}{OT1}{pzc}{m}{it} %% to make \mathpzc typeset its argument in Zapf Chancery (see page 16 of "The Great, Big List of LATEX Symbols" by David Carlisle, Scott Pakin, & Alexander Holt (2001)) %%

\newcommand{\chisq}{\mathpzc{\chi^{2}}}

\newcommand{\sq}{^{2}}

\newcommand{\df}{\mathpzc{df}} %% degrees of freedom (df) %%

%
% ----------------------------- %
% Command to insert "ToDo" tags
% ----------------------------- %

\newcommand{\todo}{
    \textsuperscript{
        \tiny{
            \fcolorbox{dkred}{black!10}{
                \color{red}{
                    \textbf{\texttt{[ToDo]}}
                }
            }
        }
    }
}

\newcommand{\inprogress}{
    \textcolor{blue}{
        \textbf{\textit{\texttt{[In Progress]}}}
    }
}

\newcommand{\complete}{
    \sout{
        \textcolor{slgray}{
            \textit{\texttt{[Complete]}}
        }
    }
}

\newcommand{\edit}[1]{
    \textcolor{red}{
        \texttt{#1}
    }
}

\newcommand{\todot}{
    \textcolor{red}{\Large{$\mathbf{^{\otimes}}$}}
}

\usepackage{wrapfig}

\begin{document}

\maketitle



{
\setcounter{tocdepth}{1}
\tableofcontents
}

\newpage

\centerline{\Large{\textsc{S3 - IPV Interventions \& Evaluations}}}

\section{\texorpdfstring{\textcolor[HTML]{5b0057}{rumptz1991ecological}}{}}\label{section}

\textsc{\large{An ecological approach to tracking battered women over time}}
(\emph{Violence and Victims})

``Examined the feasibility of using an experimental, longitudinal design
to determine the effects of an advocacy program designed to increase
battered women's access to community resources. The current research
employed a multitude of techniques to follow 139 battered women over the
1st yr following their stay at a shelter for women with abusive
partners. The tracking rate was very successful; 96\% were found and
interviewed at the 10-wk project termination point, 96\% at the 6-mo
follow-up, and 94\% at the 12-mo follow-up. The intricate system of
procedures used for successfully tracking this mobile population over
time are presented as an effective methodology for doing necessary
longitudinal research with battered women as well as other transient or
difficult to follow populations.''

\section{\texorpdfstring{\textcolor[HTML]{5b0057}{gondolf1999comparison}}{}}\label{section-1}

\textsc{\large{A comparison of four batterer intervention systems: Do court referral, program length, and services matter?}}
(\emph{Journal of Interpersonal Violence})

``Acomparative multisite evaluation was conducted in four geographically
distributed cities to examine the relative effectiveness of different
approaches to batterer intervention. The intervention systems represent
a range of court-referral procedures (pretrial or postconviction),
program duration (3 mo to 9 mo), and additional services (occasional
referral or in-house alcohol treatment). At each site, 210 men (mean age
32 yrs) were recruited and assessed using the Millon Clinical Multiaxial
Inventory and Michigan Alcoholism Screening Test. The batterers'
partners were interviewed by phone every 3 mo over a 15-mo follow-up
after intake, with a response rate of 77\% overall. There was no
significant difference in the reassault rate, portion of men making
threats, and victim quality of life across the four sites. The longest,
most comprehensive program did, however have a significantly lower rate
of severe reassault substantiated in a logistic regression controlling
background variables. The findings suggest that differing intervention
systems that conform to fundamental standards can achieve similar
outcomes.''

\section{\texorpdfstring{\textcolor[HTML]{5b0057}{thompson2000identification}}{}}\label{section-2}

\textsc{\large{Identification and management of domestic violence: A randomized trial}}
(\emph{American Journal of Preventive Medicine})

``Examined the effectiveness of an intensive intervention program to
increase asking about domestic violence (DV) and improve the
identification of and assistance for DV victims. Adult care team members
of 2 primary care clinics received intervention training, with opinion
leaders receiving extra training sessions. Ss completed questionnaires
concerning provider knowledge, attitudes, and beliefs at baseline and at
9--10 mo and 21--23 mo follow-ups. Level of care was measured by
reviewing medical records at baseline and 9 mo. Results show that at 9
mo, scores for self-efficacy, fear of offense, safety concerns, and
perceived asking about DV improved in intervention Ss. Scores for
self-efficacy, fear of offense, and safety concerns remained high at
21--23 mo follow-up. Overall, asking about DV increased but the recorded
quality of DV patient assistance did not change. Findings suggest that
DV identification can be improved by including screening questions on
physical examination questionnaires, placing posters in patient care
areas, and displaying DV brochures in restrooms.''

\section{\texorpdfstring{\textcolor[HTML]{5b0057}{gregory2002effects}}{}}\label{section-3}

\textsc{\large{The effects of batterer intervention programs: The battered women's perspectives}}
(\emph{Violence Against Women})

``Examined the perspectives of battered women, whose spouses or partners
have been court-ordered to participate in a batterer intervention
program, on the program's effects on their partners, themselves, and
their families. Through in-depth interviews, 33 women (average age 36.5
yrs) described their experiences, expectations, and feelings before,
during, and after their partner participated in the program. The
interviewees also discussed the impact of the program on the batterers'
behavior and their own lives. The results are seen to shed light on the
effects of program participation on batterers' behavior and the way in
which referral and program participation affect their female partners.
It is concluded that the results underscore the value of incorporating
battered women's perspectives and experiences in evaluating the effects
of batterer intervention programs and designing their service
delivery.''

\section{\texorpdfstring{\textcolor[HTML]{5b0057}{sullivan2002findings}}{}}\label{section-4}

\textsc{\large{Findings from a community-based program for battered women and their children}}
(\emph{Journal of Interpersonal Violence})

``The effectiveness of a strengths- and community-based support and
advocacy intervention for battered women and their children was
examined. The study included a longitudinal, experimental design and
employed multimethod strategies to measure children's (aged 6.5-11 yrs
old) exposure to abuse and their self-competence over a period of 8 mo.
Maternal experience of abuse and maternal well-being were also assessed.
The experimental intervention involved advocacy for mothers and their
children and a 10-wk support and education group for the children.
Families in the experimental condition received the free services of a
trained paraprofessional for 6-8 hrs per wk over 16 wks. 80 (mean age 31
yrs old) mothers and their 80 children participated in the study.
Findings were modest but promising. Children in the experimental
condition reported significantly higher self-competence in several
domains compared to children in the control group. The intervention
caused improvement in women's depression and self-esteem over time.
Policy, practice and research implications are discussed.''

\section{\texorpdfstring{\textcolor[HTML]{5b0057}{foshee2004assessing}}{}}\label{section-5}

\textsc{\large{Assessing the Long-Term Effects of the Safe Dates Program and a Booster in Preventing and Reducing Adolescent Dating Violence Victimization and Perpetration}}
(\emph{American Journal of Public Health})

``Objectives: This study determined 4-year postintervention effects of
Safe Dates on dating violence, booster effects, and moderators of the
program effects. Methods: We gathered baseline data in 10 schools that
were randomly allocated to a treatment condition. We collected follow-up
data 1 month after the program and then yearly thereafter for 4 years.
Between the 2- and 3-year follow-ups, a randomly selected half of
treatment adolescents received a booster. Results: Compared with
controls, adolescents receiving Safe Dates reported significantly less
physical, serious physical, and sexual dating violence perpetration and
victimization 4 years after the program. The booster did not improve the
effectiveness of Safe Dates. Conclusions: Safe Dates shows promise for
preventing dating violence but the booster should not be used.''

\section{\texorpdfstring{\textcolor[HTML]{5b0057}{hendricks2006recidivism}}{}}\label{section-6}

\textsc{\large{Recidivism Among Spousal Abusers: Predictions and Program Evaluation}}
(\emph{Journal of Interpersonal Violence})

``The relative effectiveness of two interventions for dealing with 200
court-referred spousal abusers is examined. The overall failure rate is
17.5\%, with most recidivism occurring during the first 6 months after
treatment. Offenders who completed a 14-week group treatment program
called SAFE manifest significantly lower rates of recidivism (10.6\%)
than do offenders who did not complete the mandated treatment (38.8\%).
Some high-risk clients are referred to a cognitive restructuring
treatment program called R\&R, and those completing both programs
(despite their high-risk status) have a recidivism rate of only 23.5\%.
Prediction of recidivism is difficult, with the LSI-R scores correctly
predicting only 66\% of the outcomes, using a cut score of 11.5. The
exploration of other predictors is encouraged.''

\section{\texorpdfstring{\textcolor[HTML]{5b0057}{hovell2006evaluation}}{}}\label{section-7}

\textsc{\large{Evaluation of a Police and Social Services Domestic Violence Program: Empirical Evidence Needed to Inform Public Health Policies}}
(\emph{Violence Against Women})

``The Family Violence Response Team (FVRT) responded to police calls for
domestic violence and provided services to victims. Police records were
followed for (a) 327 FVRT clients with an index police visit in 1998 and
(b) 498 nonconcurrent controls with an index visit in 1997. Except for
marriage, no demographic characteristics were associated with batterer
recidivism, as measured by police calls. The between-group odds ratio
(OR) suggested that FVRT clients experienced a 1.7 (95\% Confidence
Interval CI{]}: 1.2 to 2.5) times greater recidivism rate than controls.
Although increased reporting cannot be ruled out, results raise concerns
about the effects of domestic violence interventions.''

\section{\texorpdfstring{\textcolor[HTML]{5b0057}{silvergleid2006how}}{}}\label{section-8}

\textsc{\large{How Batterer Intervention Programs Work: Participant and Facilitator Accounts of Processes of Change}}
(\emph{Journal of Interpersonal Violence})

``Understanding what facilitates change in men who perpetrate domestic
violence can aid the development of more effective batterer intervention
programs (BIPs). To identify and describe key change processes, in-depth
interviews were conducted with nine successful BIP completers and with
10 intervention group facilitators. The accounts described a range of
individual-level processes of change consistent with prior research but
also included several processes spanning the community, organizational,
and group levels of analysis. Program completers and facilitators gave
mostly similar accounts, though differed in their emphasis of criminal
justice system sanctions, group resocialization of masculinity, and the
participants' own decision to change. All accounts especially emphasized
group-level processes and the importance of balancing support and
confrontation from facilitators and group members. The findings
demonstrate the importance of obtaining multiple perspectives on change
processes, and support ecological and systems models of batterer
intervention.''

\section{\texorpdfstring{\textcolor[HTML]{5b0057}{contrino2007compliance}}{}}\label{section-9}

\textsc{\large{Compliance and learning in an intervention program for partner-violent men}}
(\emph{Journal of Interpersonal Violence})

``Although research has yielded mixed findings regarding the
effectiveness of intervention programs for partner-violent men, it
appears that greater participant compliance with such programs is
associated with better outcomes. However, no research to date has
jointly examined compliance with intervention programs and the extent to
which partner-violent men learn specific information presented during
the programs. The current study makes use of existing data to evaluate
general and specific elements of partner-violent men's compliance with
(i.e., active, appropriate participation in) an intervention program and
recall of key points from the program. Results from a subsample of 22
men indicate that at program termination, those rated as having
been''process conscious" during intervention group sessions, having
self-disclosed during sessions, having evidenced awareness and use of
techniques to avoid violence, and having used respectful language show
greater recall of material taught in the program. This finding points to
the potential benefit of taking steps to increase men's active
participation in programs and of studying active engagement as a
mediator of program effects on men's violence toward partners."

\section{\texorpdfstring{\textcolor[HTML]{5b0057}{muftic2007evaluation}}{}}\label{section-10}

\textsc{\large{An Evaluation of Gender Differences in the Implementation and Impact of a Comprehensive Approach to Domestic Violence}}
(\emph{Violence Against Women})

``The primary goal of society's response to domestic violence is the
protection of the victim from further abuse. Recently, the coordinated
community response (CCR) has been developed as one example of an
approach aimed at reaching this goal. Prior research has generally found
support for the model, with male offenders recidivating at lower rates.
The current study examines whether a comprehensive, community-based
approach is capable of reducing recidivism rates among male and female
offenders. Comparisons are made between 70 female and 131 male
offenders. Specific attention is given to the intervention process,
including differences in service or treatment component completion and
recidivism by gender.''

\section{\texorpdfstring{\textcolor[HTML]{5b0057}{gillum2008benefits}}{}}\label{section-11}

\textsc{\large{The benefits of a culturally specific intimate partner violence intervention for African American survivors}}
(\emph{Violence Against Women})

``In light of evidence and theorization of culturally specific factors
contributing to intimate partner violence (IPV) within African American
relationships and the Eurocentric approach many mainstream agencies take
to service delivery, researchers have indicated a need for culturally
appropriate IPV interventions for African American survivors to
adequately address the issue of IPV within this community. The purpose
of the current study was to qualitatively investigate how helpful a
culturally specific IPV program, which targets the African American
community, has been to African American female survivors. Results
suggest that this culturally specific agency is successfully meeting the
needs of these survivors.''

\section{\texorpdfstring{\textcolor[HTML]{5b0057}{roffman2008mens}}{}}\label{section-12}

\textsc{\large{The men's domestic abuse check-up: A protocol for reaching the nonadjudicated and untreated man who batters and who abuses substances}}
(\emph{Violence Against Women})

``Batterer intervention programs primarily work with individuals
mandated to participate. Commonly, attrition is high and outcomes are
modest. Motivational enhancement therapy (MET), most widely studied in
the substance abuse field, offers a potentially effective approach to
improving self-referral to treatment, program retention, treatment
compliance, and posttreatment outcomes among men who batter and who
abuse substances. A strategy for using a catalyst variant of MET
(a''check-up``) to reach untreated, nonadjudicated perpetrators is
described in detail. Unique challenges in evaluating the success of this
approach are discussed, including attending to victim safety and
determining indicators of increased motivation for change.''

\section{\texorpdfstring{\textcolor[HTML]{5b0057}{price2009batterer}}{}}\label{section-13}

\textsc{\large{Batterer intervention programs: A report from the field}}
(\emph{Violence and Victims})

``Over the past 25 years, batterer intervention has become the most
probable disposition following a plea or conviction on domestic battery
charges and, consequently, batterer intervention programs (BIPs) have
proliferated. Despite their popularity, and recent attempts by states to
regulate practice, little is known about the actual programs operating
in the field. The aim of this study was to examine the philosophy,
structure, leadership, curricula, and support systems of BIPs.
Respondents from 276 batterer intervention programs in 45 states
described their programs via an anonymous, Web-based survey. The results
provide some insight regarding the workings of actual BIPs and also
point out problems such as the dearth of programs in languages other
than English and the failure to translate recommendations for
prescriptive approaches into practice.''

\section{\texorpdfstring{\textcolor[HTML]{5b0057}{enriquez2010development}}{}}\label{section-14}

\textsc{\large{Development and feasibility of an HIV and IPV prevention intervention among low-income mothers receiving services in a Missouri day care center}}
(\emph{Violence Against Women})

``This article outlines the development and feasibility of an HIV and
IPV prevention intervention. Researchers formed a partnership with a
group of women representative of the population that the intervention
was intended to reach using methods derived from participatory action
research. The use of health protective behaviors changed from pre- to
postintervention in the clinically desirable direction. Results
indicated that intervention delivery was feasible in the novel setting
of a large urban day care center. This intervention has promise as a
strategy to reduce HIV among low-income women; however, a controlled
study is indicated to further examine intervention efficacy.''

\section{\texorpdfstring{\textcolor[HTML]{5b0057}{welland2010culturally}}{}}\label{section-15}

\textsc{\large{Culturally specific treatment for partner-abusive Latino men: A qualitative study to identify and implement program components}}
(\emph{Violence and Victims})

``Research based on a demographic survey and qualitative interviews of
Latino intimate partner violence perpetrators in Southern California
forms the basis of a Spanish-language treatment program designed to be
culturally appropriate for Latino immigrant men, and piloted for 4 years
with their input. Culturally-specific topics emphasized by participants
and integrated into the program are: effective parenting skills for men;
gender roles; discussion of discrimination towards immigrants and women;
immigration and changing gender roles; marital sexual abuse; and
spirituality as related to violence prevention. Attention is given to
alcohol abuse and childhood trauma. Results suggest the desirability of
an empathic and culturally-sensitive approach, without diminishing
responsibility. This program was designed to help clinicians refine
their skills and effectiveness in working with this rapidly expanding
population.''

\section{\texorpdfstring{\textcolor[HTML]{5b0057}{feder2011need}}{}}\label{section-16}

\textsc{\large{The need for experimental methodology in intimate partner violence: Finding programs that effectively prevent IPV}}
(\emph{Violence Against Women})

``The lack of rigorous evaluations of intimate partner violence (IPV)
programs has severely limited our knowledge about what works. However,
IPV programs can be rigorously evaluated through randomized controlled
trials (RCTs) conducted ethically and safely. This article provides an
example of how a RCT to test an IPV preventive intervention---the
Enhanced Nurse Family Partnership Study (ENFPS)---was successfully
implemented by a partnership of researchers and practitioners. The
article concludes with some recommendations, arrived at by the
researchers and practitioners on the ENFPS team, for achieving a
successful collaboration thought to be essential in executing a field
experiment.''

\section{\texorpdfstring{\textcolor[HTML]{5b0057}{potter2011bringing}}{}}\label{section-17}

\textsc{\large{Bringing in the target audience in bystander social marketing materials for communities: Suggestions for practitioners}}
(\emph{Violence Against Women})

``The Know Your Power TM social marketing campaign images model active
bystander behaviors that target audience members can use in situations
where sexual and relationship violence and stalking are occurring, have
occurred, or have the potential to occur. In this practitioner note, we
describe strategies that we have used to engage target audience members
in the development of the social marketing campaign that we hope can be
used by practitioners. We give examples from the development and
evaluation of the Know Your Power TM social marketing campaign that used
focus group and other types of feedback from the target audience to
inform the direction of the campaign.''

\section{\texorpdfstring{\textcolor[HTML]{5b0057}{boal2014barriers}}{}}\label{section-18}

\textsc{\large{Barriers to compliance with Oregon Batterer Intervention Program standards}}
(\emph{Violence and Victims})

``Although standards for batterer intervention programs (BIPs) have been
adopted in nearly all U.S. states, there is no evidence that standards
are implemented and no information about challenges programs may
encounter in efforts to comply with standards. This study uses
qualitative survey data from BIPs in the state of Oregon ( N = 42) to
identify barriers to implementation during a 2-year period following the
introduction of state standards. Nine challenges were identified
including difficulty finding qualified facilitators, inadequate funding,
difficulty meeting training requirements, high workloads, trouble
creating and maintaining collaborations, inability to accommodate
diverse participant needs, conflict between state standards and county
requirements, and perceived gaps between standards and evidence-based
practices. These findings inform controversy surrounding BIP standards
and efforts to increase BIP effectiveness.''

\section{\texorpdfstring{\textcolor[HTML]{5b0057}{boal2014impact}}{}}\label{section-19}

\textsc{\large{The impact of legislative standards on batterer intervention program practices and characteristics}}
(\emph{American Journal of Community Psychology})

``Changes in social policy are often pursued with the goal of reducing a
social problem by improving prevention efforts, intervention program
practices, or participant outcomes. State legislative standards for
intimate partner violence intervention programs have been adopted nearly
universally across the US, however, we do not know whether such
standards actually achieve the intended goal of affecting programs'
policies and practices. To assess the effect that batterer intervention
program (BIP) standards have on policies and practices of programs, this
study used longitudinal surveys collected as part of an ongoing
evaluation conducted from 2001 to the present to compare intervention
program ( N = 74) characteristics and practices at three time points
before and after the adoption of standards in Oregon. Analyses were
conducted to examine all BIPs in Oregon at each time point, as well as
change among a subset of programs in existence at all survey
assessments. Results indicate that across all programs, the use of mixed
gender group co-facilitation increased by 14 \% between 2004 and 2008,
while program length increased by approximately 12 weeks. However, other
practices such as programs' coordination with community partners were
unchanged. Analyses of within-program change revealed fewer differences,
with only program length increasing significantly over the three
assessments. These and other findings indicate that while standards
affected program length as intended, other practices commonly addressed
by legislative standards remained unchanged. The findings provide needed
information regarding programs' compliance with components of the
standards, the potential need for compliance monitoring, and the
potential impact of state standards on program effectiveness and on the
prevalence of intimate partner violence.''

\section{\texorpdfstring{\textcolor[HTML]{5b0057}{ermentrout2014this}}{}}\label{section-20}

\textsc{\large{'This is about me': Feasibility findings from the children's component of an IPV intervention for justice-involved families}}
(\emph{Violence Against Women})

``Two community-based agencies collaborated to create a program for
justice-involved female intimate partner violence (IPV) survivors and
their children. Our research team conducted a feasibility study of the
children's program using an exploratory, multimethod qualitative design
with child participants ( n = 8), adult participants ( n = 18), and
providers ( n = 7). Analyses determined four key findings: (a)
importance of attendance; (b) the need for a flexible, child-driven
curriculum; (c) improvement through expression and peer bonding; and (d)
the value of specific program content. The findings point to
indispensable program elements and enhancement recommendations.
Implications for other communities and providers serving IPV-exposed
children are described.''

\section{\texorpdfstring{\textcolor[HTML]{5b0057}{fox2015development}}{}}\label{section-21}

\textsc{\large{Development of the Attitudes to Domestic Violence questionnaire for children and adolescents}}
(\emph{Journal of Interpersonal Violence})

``To provide a more robust assessment of the effectiveness of a domestic
abuse prevention education program, a questionnaire was developed to
measure children's attitudes to domestic violence. The aim was to
develop a short questionnaire that would be easy to use for
practitioners but, at the same time, sensitive enough to pick up on
subtle changes in young people's attitudes. We therefore chose to ask
children about different situations in which they might be willing to
condone domestic violence. In Study 1, we tested a set of 20 items,
which we reduced by half to a set of 10 items. The factor structure of
the scale was explored and its internal consistency was calculated. In
Study 2, we tested the factor structure of the 10-item Attitudes to
Domestic Violence (ADV) Scale in a separate calibration sample. Finally,
in Study 3, we then assessed the test--retest reliability of the 10-item
scale. The ADV Questionnaire is a promising tool to evaluate the
effectiveness of domestic abuse education prevention programs. However,
further development work is necessary.''

\section{\texorpdfstring{\textcolor[HTML]{5b0057}{howell2015strengthening}}{}}\label{section-22}

\textsc{\large{Strengthening positive parenting through intervention: Evaluating the moms' empowerment program for women experiencing intimate partner violence}}
(\emph{Journal of Interpersonal Violence})

``This study examined the effectiveness of an evidence-based
intervention in changing the positive and negative parenting practices
of 120 mothers who experienced intimate partner violence (IPV) in the
last 2 years. Mothers assigned to the treatment group participated in a
10-session evidence-based intervention, known as the Moms' Empowerment
Program, which targets the mental health problems of women and works to
increase access to resources and improve parenting abilities of women
exposed to IPV. Participants were interviewed at baseline and
immediately following the intervention or waitlist period, representing
an elapsed time of approximately 5 weeks. After controlling for relevant
demographic variables, violence severity, and mental health, women
showed significantly more change in their positive parenting scores if
they were in the treatment condition. No significant differences were
found between the treatment and comparison groups in their negative
parenting practices change scores. These findings suggest that even
short-term intervention can improve positive parenting skills and
parenting knowledge for women who have experienced partner abuse.''

\newpage

\centerline{\Large{\textsc{S4 - Female Same-Sex/Same-Gender Intimate Partner Violence}}}

\section{\texorpdfstring{\textcolor[HTML]{5b0057}{lockhart1994letting}}{}}\label{section-23}

\textsc{\large{Letting out the secret: Violence in lesbian relationships}}
(\emph{Journal of Interpersonal Violence})

``Examined the extent, nature, and correlates of conflict and violence
in lesbian relationships. Survey responses of 284 lesbians (aged 21--60
yrs) suggest that lesbian violence was not rare. 90\% of Ss had been
recipients of one or more acts of verbal aggression from their intimate
partners during the year prior to this study. These acts tended to
revolve around conflicts about partner's job, partner's emotional
dependency, money, housekeeping/cooking duties, sexual activities, and
use of alcohol/drugs. 31\% of the Ss reported one or more incidents of
physical abuse. Physical abuse was triggered by or erupted around issues
of power imbalance and/or a struggle for varying levels of
interdependency and autonomy in the relationship. Ss who perceived that
their partners felt less of a need for social fusion in the relationship
reported lower levels of verbal aggression/abuse.''

\section{\texorpdfstring{\textcolor[HTML]{5b0057}{wise1997comparison}}{}}\label{section-24}

\textsc{\large{Comparison of beginning counselors' responses to lesbian vs heterosexual partner abuse}}
(\emph{Violence and Victims})

``This study compared responses of master and doctoral level counseling
students to 2 domestic violence scenarios. Participants read a 2
paragraph description of a battering incident involving either a
heterosexual or lesbian couple and then gave their impressions via a
series of open and closed ended questions. Scenarios were identical save
the manipulation of sexual partner as same or opposite sex. Experience
and/or education with battered and/or gay/lesbian clients is also
examined. Results indicated that Ss perceived the heterosexual battering
incident as more violent than the lesbian battering incident and would
be more likely to charge the male batterer than the female batterer with
assault. Differences in treatment recommendations were made according to
sexual orientation of the victim. Less than half or the respondents had
coursework or practical experience pertaining to domestic violence
and/or gay/lesbian concerns.''

\section{\texorpdfstring{\textcolor[HTML]{5b0057}{bernhard2000physical}}{}}\label{section-25}

\textsc{\large{Physical and sexual violence experienced by lesbian and heterosexual women}}
(\emph{Violence Against Women})

``Explored whether there are differences in the violence experienced by
lesbian and heterosexual women and in the actions used in response to
violence. A convenience sample of 136 lesbian and 79 heterosexual women
(all Ss aged 19--67 yrs) completed survey questionnaires. Significantly
more lesbians (51\%) than heterosexual women (33\%) had experienced
nonsexual physical violence, and there was no difference between the
groups in the prevalence of sexual violence (lesbian 54\%, heterosexual
44\%). The principal actions for all women in response to violence were
avoidance, talking to someone, and doing nothing---passive strategies
that have limited value.''

\section{\texorpdfstring{\textcolor[HTML]{5b0057}{giorgio2002speaking}}{}}\label{section-26}

\textsc{\large{Speaking silence: Definitional dialogues in abuse lesbian relationships}}
(\emph{Violence Against Women})

``Long-term, in-depth interviews with 11 abused lesbians and 10 domestic
violence advocates reveal how lesbian victims struggle to define the
relationship's abuse, their lesbian identity, and their own
understanding of gendered violence in the context of cultural and
institutional stigmatization of lesbians. By understanding abused
lesbians' silence as constitutive of their definitional dialogues about
their relationships and the abuse, researchers and advocates can begin
to determine who asserts definitional hegemony in the relationship. The
author concludes by suggesting practical strategies that researchers and
advocates can deploy to include abused lesbians in domestic violence
theory, praxis, and services.''

\section{\texorpdfstring{\textcolor[HTML]{5b0057}{younglove2002law}}{}}\label{section-27}

\textsc{\large{Law enforcement officers' perceptions of same sex domestic violence - Reason for cautious optimism}}
(\emph{Journal of Interpersonal Violence})

``In 1994, California amended its domestic violence legislation to
include same sex couples. It is commonly believed within the gay and
lesbian community that homophobia induces law enforcement officers to
respond differently to incidents of domestic violence involving same sex
couples than to incidents involving opposite sex couples. However, the
dearth of empirical research has precluded assessing the validity of
this assumption, especially since the legal mandate has changed. This
study addressed whether the assumption of homophobia among police
officers is supported through a survey designed to ascertain perceptions
about same sex domestic violence. Results showed no differences in how
police perceived a scenario of domestic violence based on the sexual
orientation of the involved couple. To the extent that expressed
perception reflects acknowledging a need to comply with the legal
mandate, there is reason to be hopcful that hornophobia need not
deterappropriate /ou, entowetnent response to theproblem of domestic
violence among same sex couples.''

\section{\texorpdfstring{\textcolor[HTML]{5b0057}{fortunata2003demographic}}{}}\label{section-28}

\textsc{\large{Demographic, psychosocial, and personality characteristics of lesbian batterers}}
(\emph{Violence and Victims})

``Prevalence of domestic violence (DV) in lesbian and heterosexual
relationships appears to be similar. Despite this, few studies have
examined factors associated with DV in lesbian relationships, and even
fewer have examined characteristics of lesbian batterers. Demographic
and psychosocial characteristics and personality traits were examined in
100 lesbians in current relationships (33 Batterers and 67
Nonbatterers). Results indicated that Batterers were more likely to
report childhood physical and sexual abuse and higher rates of alcohol
problems. Results from the MCMI-III indicated that, after controlling
for Debasement and Desirability indices, Batterers were more likely to
report aggressive, antisocial, borderline, and paranoid personality
traits, and higher alcohol-dependent, drug-dependent, and delusional
clinical symptoms compared to Nonbatterers. These results provide
support for social learning and psychopathology theoretical models of DV
and clinical observations of lesbian batterers, and expand our current
DV paradigms to include information about same-sex DV.''

\section{\texorpdfstring{\textcolor[HTML]{5b0057}{mccloskey2003contribution}}{}}\label{section-29}

\textsc{\large{The contribution of marital violence to adolescent aggression across different relationships}}
(\emph{Journal of Interpersonal Violence})

``Different forms of aggression were measured in 296 young men and women
participating in a study dating from their childhood that included
families with marital violence. The youth reported on their perpetration
of physical aggression with same-sex peers, dating partners, and
parents. Measures were also collected on youth depression and empathy.
Childhood exposure to marital violence predicted aggression toward peers
for all youth. Marital violence was also related to child-to-parent
aggression but only for youth older than 18. Youth from maritally
violent homes were more likely to be depressed as adolescents. Elevated
depression partially mediated the impact of marital violence on peer
aggression and was associated with dating aggression among girls.
Although marital violence in childhood was unrelated to empathy scores
in adolescence, empathic youth were less likely to engage in dating
aggression and peer aggression. Findings indicate that further emphasis
should be placed on mental health problems and empathy building in youth
exposed to marital violence.''

\section{\texorpdfstring{\textcolor[HTML]{5b0057}{glass2004female-perpetrated}}{}}\label{section-30}

\textsc{\large{Female-perpetrated femicide and attempted femicide - A case study}}
(\emph{Violence Against Women})

``Femicide, the homicide of women, is the seventh leading cause of
premature death for women overall. Intimate partner (IP) homicide
accounts for approximately 40\% to 50\% of U.S. femicides. The vast
majority of IP femicides are perpetrated by male partners, with .05\% of
IP femicides in the U.S. perpetrated by female partners. Few studies
have examined intimate partner violence (IPV) between female partners
and no study (to the authors' knowledge) has examined female-perpetrated
IP femicide and attempted femicide in same-sex relationships. This case
study examines IP femicide and attempted femicide among a small Sample
of women in same-sex relationships. The findings call attention to this
important women's health issue, expand our contextual understanding of
violence in female same-sex relationships, and assist health care, law
enforcement, judiciary, service, and advocacy professionals to develop
prevention strategies and resources to reduce the risk of serious injury
and death among women in same-sex relationships.''

\section{\texorpdfstring{\textcolor[HTML]{5b0057}{balsam2005relationship}}{}}\label{section-31}

\textsc{\large{Relationship Quality and Domestic Violence in Women's Same-Sex Relationships: The Role of Minority Stress}}
(\emph{Psychology of Women Quarterly})

``Despite a large body of literature addressing relationship quality and
domestic violence in women's same-sex relationships, few studies have
empirically examined how stress specific to living as a lesbian or
bisexual woman might correlate with these relationship variables. Degree
of outness, internalized homophobia, lifetime and recent experiences of
discrimination, butch/femme identity, relationship quality, and lifetime
and recent experiences of domestic violence were assessed in a sample of
272 predominantly European American lesbian and bisexual women. Lesbian
and bisexual women were found to be comparable on most relationship
variables. In bivariate analyses, minority stress variables
(internalized homophobia and discrimination) were associated with lower
relationship quality and both domestic violence perpetration and
victimization. Outness and butch/femme identity were largely unrelated
to relationship variables. Path analysis revealed that relationship
quality fully mediated the relationship between internalized homophobia
and recent domestic violence.''

\section{\texorpdfstring{\textcolor[HTML]{5b0057}{sorenson2005restraining}}{}}\label{section-32}

\textsc{\large{Restraining orders in California - A look at statewide data}}
(\emph{Violence Against Women})

``The authors tabulated statewide administrative data for all types of
restraining orders. On June 6, 2003, there were 227,941 active
restraining orders against adults in California; most were for domestic
violence. Rates of restraining orders (i.e., restrained persons) were
highest for men, African Americans, and 25- to 34-year-olds. In 72.2\%
of the orders, a woman was to be protected and a man was to be
restrained, in 19.3\%, the restrained and protected persons were of the
same sex. Although state law prohibits the purchase or possession of a
firearm by persons against whom a restraining order is issued, 9.2\% of
the orders documented no firearm restrictions.''

\section{\texorpdfstring{\textcolor[HTML]{5b0057}{heintz2006intimate}}{}}\label{section-33}

\textsc{\large{Intimate Partner Violence and HIV/STD Risk Among Lesbian, Gay, Bisexual, and Transgender Individuals}}
(\emph{Journal of Interpersonal Violence})

``To date, there has been little research examining HIV/STD risk among
lesbian, gay, bisexual, and transgender (LGBT) individuals who are in
abusive relationships. This article uses data collected from a
community-based organization that provides counseling for LGBT victims
of intimate partner violence (IPV). A total of 58 clients completed the
survey, which inquired as to sexual violence and difficulties
negotiating safer sex with their abusive partners. A large percentage of
participants reported being forced by their partners to have sex (41\%).
Many stated that they felt unsafe to ask their abusive partners to use
safer sex protection or that they feared their partners' response to
safer sex (28\%). In addition, many participants experienced sexual
(19\%), physical (21\%), and/or verbal abuse (32\%) as a direct
consequence of asking their partner to use safer sex protection.
Training counselors on issues of sexuality and safer sex will benefit
victims of IPV.''

\section{\texorpdfstring{\textcolor[HTML]{5b0057}{pattavina2007comparison}}{}}\label{section-34}

\textsc{\large{A comparison of the police response to heterosexual versus same-sex intimate partner violence}}
(\emph{Violence Against Women})

``It has been argued that the police do not respond to domestic calls
involving same-sex couples in the same manner as they respond to calls
involving heterosexual couples. A major problem facing researchers
examining the police response to cases involving same-sex couples has
been the lack of adequately sized samples. In this article, the authors
utilize the 2000 National Incident Based Reporting System database,
which contains 176,488 intimate partner assaults and intimidation
incidents reported to 2,819 police departments in 19 states. The key
issue examined is whether similar cases involving same-sex and
heterosexual couples result in the same police response.''

\section{\texorpdfstring{\textcolor[HTML]{5b0057}{bossarte2008clustering}}{}}\label{section-35}

\textsc{\large{Clustering of adolescent dating violence, peer violence, and suicidal behavior}}
(\emph{Journal of Interpersonal Violence})

``To understand the co-occurrence of multiple types of violence, the
authors developed a behavioral typology based on self-reports of
suicidal behaviors, physical violence, and psychological abuse. Using a
sample of dating adolescents from a high-risk school district, they
identified five clusters of behaviors among the 1,653 students who
reported being abusive or violent in the past year. Victimization and
perpetration with same-sex peers and dating partners clustered together
among the students who reported the highest levels of abusive (n = 357)
or violent behavior (n = 146). These students also reported high levels
of suicidal behavior. There were few significant demographic differences
across clusters. The implications of the results for the need to design
and evaluate efforts to prevent multiple types of violence are
discussed.''

\section{\texorpdfstring{\textcolor[HTML]{5b0057}{glass2008risk}}{}}\label{section-36}

\textsc{\large{Risk for reassault in abusive female same-sex relationships}}
(\emph{American Journal of Public Health})

``Objectives. We revised the Danger Assessment to predict reassault in
abusive female same-sex relationships. Methods. We used focus groups and
interviews to evaluate the assessment tool and identify new risk factors
and telephone interviews at baseline and at 1-month follow-up to
evaluate the revised assessment. Results. The new assessment tool
comprised 8 original and 10 new items. Predictors included increase in
physical violence (relative risk ratio \{{[}\}RRR{]} = 1.95; 95\%
confidence interval \{{[}\}CI{]}=0.84, 4.54), constant jealousy or
possessiveness of abuser (RRR = 4.07; 95\% CI = 0.61, 27.00),
cohabitation (RRR = 1.96; 95\% CI = 0.54, 7.12), threats or use of gun
by abuser (RRR=1.93; 95\% 0=039, 4.75), alcoholism or problem drinking
of abuser (RRR = 1.47; 95\% CI =0.79, 2.71), illegal drug use or abuse
of prescription medications by abuser (RRR = 1.33; 95\% CI = 0.72,
2.46), stalking by abuser (RRR = 1.39; 95\% CI = 0.70, 2.76), failure of
individuals to take victim seriously when she sought help (RRR 1.66;
95\% CI 0.90, 3.05), victim's fear of reinforcing negative stereotypes
(RRR 1.42; 95\% CI 0.73, 2.77), and secrecy of abuse (RRR=1.72; 95\%
CI=0.74, 3.99). Both unweighted (P \textless{}.005) and weighted (P
\textless{}.004) versions of the revised assessment were significant
predictors of reassault. Conclusions. The revised Danger Assessment
accurately assesses risk of reassault in abusive female relationships.''

\section{\texorpdfstring{\textcolor[HTML]{5b0057}{hassouneh2008influence}}{}}\label{section-37}

\textsc{\large{The influence of gender role stereotyping on women's experiences of female same-sex intimate partner violence}}
(\emph{Violence Against Women})

``Female same-sex intimate partner violence (FSSIPV) is a serious
problem that affects the health and safety of lesbian and bisexual
women. To begin to address the paucity of research, a mixed methods
study was conducted to identify shared and unique risk and protective
factors for FSSIPV. This article reports on qualitative findings related
to the influence of gender role stereotyping on women's experiences of
FSSIPV Findings indicate that gender role stereotyping shapes women's
experiences of FSSIPV by influencing individual, familial, community,
and societal perceptions and responses to this phenomenon.''

\section{\texorpdfstring{\textcolor[HTML]{5b0057}{swahn2008measuring}}{}}\label{section-38}

\textsc{\large{Measuring sex differences in violence victimization and perpetration within date and same-sex peer relationships}}
(\emph{Journal of Interpersonal Violence})

``This study examines sex differences in the patterns of repeated
perpetration and victimization of physical violence and psychological
aggression within dating relationships and same-sex peer relationships.
Data were obtained from the Youth Violence Survey: Linkages among
Different Forms of Violence, conducted in 2004, and administered to all
public school students enrolled in grades 7, 9, 11 and 12 (N = 4,131) in
a high-risk school district. Analyses of adolescents who dated in the
past year (n = 2,888) show that girls are significantly more likely than
boys to report physical violence and psychological aggression
perpetration within dating relationships. However, boys are
significantly more likely than girls to report physically injuring a
date. Boys are also significantly more likely than girls to report
physical violence victimization and perpetration within same-sex peer
relationships. Implications and directions for future research are
discussed.''

\section{\texorpdfstring{\textcolor[HTML]{5b0057}{blosnich2009comparisons}}{}}\label{section-39}

\textsc{\large{Comparisons of intimate partner violence among partners in same-sex and opposite-sex relationships in the United States}}
(\emph{American Journal of Public Health})

``Using 2005--2007 Behavioral Risk Factor Surveillance System data, we
examined intimate partner violence (IPV) by same-sex and opposite-sex
relationships and by Metropolitan Statistical Area status. Same-sex
victims differed from opposite-sex victims in some forms of IPV
prevalence, and urban same-sex victims had increased odds of poor
self-perceived health status (adjusted odds ratio = 2.41; 95\%
confidence interval= 1.17, 4.94). Same-sex and opposite-sex victims
experienced similar poor health outcomes, underscoring the need both of
inclusive service provision and consideration of sexual orientation in
population-based research.''

\section{\texorpdfstring{\textcolor[HTML]{5b0057}{edelen2009measurement}}{}}\label{section-40}

\textsc{\large{Measurement of Teen Dating Violence Attitudes An Item Response Theory Evaluation of Differential Item Functioning According to Gender}}
(\emph{Journal of Interpersonal Violence})

``Accurate assessment of attitudes about intimate partner violence is
important for evaluation of prevention and early intervention programs.
Assessment of attitudes about cross-gender interactions is particularly
susceptible to bias because it requires specifying the gender of the
perpetrator and the victim. As it is likely that respondents will tend
to identify with the same-gender actor, items and scales assessing
attitudes about intimate partner violence may not have equivalent
measurement properties for male and female respondents. This article
examines data from 2,575 high school students who participated in a
teen-dating violence intervention study. The majority of participants
were Latino (91\%), and the sample was nearly evenly split with respect
to gender (51\% female). Items from two scales (boy-on-girl violence, 4
items; girl-on-boy violence, 5 items) reflecting teens' attitudes about
dating violence were calibrated with the graded item response theory
(IRT) model and evaluated for differential item functioning (DIF) by
gender. A total of three items, two from the girl-on-boy violence scale
and one from the boy-on-girl violence scale, were identified as
functioning differently for girls and boys. In all cases where DIF was
detected, the item's attitudinal statement was easier to accept for the
gender group that was portrayed as victim rather than perpetrator. For
both scales, accounting for the identified DIF influenced inferences
about the magnitude of mean differences in attitudes between boys and
girls. These results support the use of IRT scores that account for DIF
to minimize measurement error and improve inferences about gender
differences in attitudes about dating violence.''

\section{\texorpdfstring{\textcolor[HTML]{5b0057}{oswald2010lesbian}}{}}\label{section-41}

\textsc{\large{Lesbian mothers' counseling experiences in the context of intimate partner violence}}
(\emph{Psychology of Women Quarterly})

``Intimate partner violence (IPV) is a significant concern for some
lesbian households with children. Yet we know of only one study that has
examined lesbian mothers' experiences with IPV. In the current study we
analyzed the counseling experiences of participants in our prior study.
Interviews with 24 lesbian mothers (12 Black, 9 White, and 3 Latina) 23
to 54 years of age ( M = 39.5) were coded using thematic analysis.
Overall, lesbian mothers experiencing IPV did seek help from counselors
( n = 15, 63\%), typically after reaching a breaking point. Counselors
were most helpful when addressing the abuse and promoting
self-empowerment, and least helpful when victim-blaming or ignoring the
abuse and/or the same-sex relationship. Lesbian mothers' perceptions
that mental health professionals were sometimes ineffective have
implications for provider training. In order to work effectively with
this population, providers should attempt to eliminate or correct
personal biases or prejudices with self-exploration and education. By
becoming more aware and knowledgeable of the nuances, struggles, and
strengths of the lesbian community, providers can gain competency in
providing therapeutic services to such clients. Mental health
professionals can also adopt an advocacy stance to assist in spreading
cultural awareness to others and support policy or institutional changes
to include same-sex IPV. Competencies can be assessed through future
studies that identify the knowledge and skills gap among mental health
professionals who frequently work with the lesbian population.''

\section{\texorpdfstring{\textcolor[HTML]{5b0057}{ackerman2011gender}}{}}\label{section-42}

\textsc{\large{The Gender Asymmetric Effect of Intimate Partner Violence on Relationship Satisfaction}}
(\emph{Violence and Victims})

``Our research examined the association between intimate partner
violence and relationship satisfaction among victims. The negative
association between victimization and relationship satisfaction was
substantially stronger for females than for males. Comparisons between
respondents reporting about same-sex relationships with those reporting
about opposite-sex relationships provided evidence that the amplified
victimization/satisfaction association among female victims is a
victim-gender effect rather than an actor-gender effect. In other words,
our findings suggest that aggression harms the quality of the intimate
partnerships of females much more so than the partnerships of males
regardless of whether a male or a female is the perpetrator. We
supplemented dialogue about the direct implications of our findings with
discussions about how these results may raise conceptual questions about
the adequacy of the instruments scholars use to study partner
aggression.''

\section{\texorpdfstring{\textcolor[HTML]{5b0057}{davidovic2011impelling}}{}}\label{section-43}

\textsc{\large{Impelling and Inhibitory Forces in Aggression: Sex-of-Target and Relationship Effects}}
(\emph{Journal of Interpersonal Violence})

``The finding of symmetry in intimate partner aggression is now
generally accepted, but the convergence of male and female rates in
these relationships remains unexplained. From qualitative analysis of
male and female focus group discussions, we identified factors believed
to influence the expression of aggression toward targets differing in
sex and degree of intimacy. These factors were then used to construct a
questionnaire in which 355 respondents indicated the applicability of
the items to conflicts with a partner, a same-sex friend, and an
opposite-sex friend. Principal component analysis revealed a clear
two-factor structure of impelling forces (tending to provoke or initiate
aggression) and inhibitory forces (tending to suppress or diminish the
likelihood of aggression). Participants' scores on scales derived from
these two factors were used in the subsequent analyses. Men reported
lower inhibition and greater impulsion toward same-sex friends than to
female friends and partners, who did not differ significantly from one
another. Women showed lower inhibition to male targets, regardless of
relationship, than to a female target. However, women rated their male
partners as significantly higher on impelling forces than their male
friends, who in turn were rated significantly higher than female
friends. The results are broadly consistent with a sex-of-target effect
corresponding to a chivalry norm held by both sexes that inhibits the
expression of aggression toward women. The reasons why women are
especially impelled to aggression by intimate partners are explored.
Disaggregating the dynamics of interpersonal conflict into impelling and
inhibitory components may prove useful in understanding and treating
dispute escalation and resolution.''

\section{\texorpdfstring{\textcolor[HTML]{5b0057}{hardesty2011lesbian/bisexual}}{}}\label{section-44}

\textsc{\large{Lesbian/bisexual mothers and intimate partner violence: Help seeking in the context of social and legal vulnerability}}
(\emph{Violence Against Women})

``Mothers in same-sex relationships face unique challenges when help
seeking for intimate partner violence (IPV). Formal helping systems
often invalidate their family relationships, leaving them vulnerable and
distrustful when help seeking. To better understand their experiences,
the authors interviewed 24 lesbian/bisexual mothers who were either in
or had left abusive same-sex relationships. Increasing severity of
violence, effects of violence on children and families, and''being
tired" influenced their definitions of the situation. Decisions to seek
formal help appeared to be influenced by their support from informal
networks and perceived stigma related to the intersection of IPV and
being lesbian or bisexual."

\section{\texorpdfstring{\textcolor[HTML]{5b0057}{iverson2011contribution}}{}}\label{section-45}

\textsc{\large{The Contribution of Childhood Family Violence on Later Intimate Partner Violence Among Robbery Victims}}
(\emph{Violence and Victims})

``This study examined the relative contributions of the three forms of
childhood family violence exposure on physical intimate partner violence
(IPV) victimization among recent robbery victims and tested a
gender-matching modeling prediction for IPV risk. Data from a sample of
103 male and 93 female victims of a robbery were analyzed to investigate
the effects of exposure to childhood physical abuse (CPA), childhood
sexual abuse (CSA), and witnessing parental violence on the likelihood
of IPV in adulthood. As expected, witnessing parental violence was
associated with a 2.4-fold increase in IPV for both men and women.
Neither CPA nor CSA was significantly associated with IPV after
accounting for the effect of witnessing parental violence. There was
support for the gender-matching hypothesis with men more likely to
report IPV if they had witnessed mother-to-father violence and women
more likely to report IPV if they had witnessed father-to-mother
violence. Witnessing parental violence is strongly associated with risk
for IPV victimization, particularly when the victim is the same-gender
parent. Future directions and clinical implications are discussed.''

\section{\texorpdfstring{\textcolor[HTML]{5b0057}{messinger2011invisible}}{}}\label{section-46}

\textsc{\large{Invisible Victims: Same-Sex IPV in the National Violence Against Women Survey}}
(\emph{Journal of Interpersonal Violence})

``With intimate partner violence (IPV) among same-sex couples largely
ignored by policy makers and researchers alike, accurately estimating
the size of the problem is important in determining whether this minimal
response is justified. As such, the present study is a secondary data
analysis of the National Violence Against Women Survey and represents
the first multiple variable regression analysis of U. S. adult same-sex
IPV prevalence using a nationally representative sample (N = 14,182).
Logistic regressions indicate that, independent of sex, respondents with
a history of same-sex relationships are more likely to experience
verbal, controlling, physical, and sexual IPV. Behaviorally''bisexual"
respondents experience the highest IPV rates and are most likely to be
victimized by an opposite-sex partner. Implications for future IPV
research regarding sexual orientation and gender are discussed."

\section{\texorpdfstring{\textcolor[HTML]{5b0057}{porter2011intimate}}{}}\label{section-47}

\textsc{\large{Intimate violence among underrepresented groups on a college campus}}
(\emph{Journal of Interpersonal Violence})

``Rape, sexual violence, psychological violence, and physical violence,
among college students have been a concern. Lifetime events are often
studied but not violence that specifically transpires while one is in
college. Underrepresented groups such as Deaf and Hard of Hearing
students, students who are gay, lesbian, and bisexual, and students who
are members of racial and ethnic minorities have not been studied as
extensively as White, heterosexual females. The authors used several
measures to investigate the incidence of sexual violence, physical and
psychological abuse among underrepresented groups in a random sample of
1,028 college students at a private, northeastern, technological campus
in upstate New York, United States and analyzed victimization rates by
gender, race/ethnicity, auditory status, and sexual orientation. Binary
logistic regression analyses found that statistically significant
differences are likely to exist between members of underrepresented
groups and groups in the majority. The study found statistically
significant associations between Deaf and Hard of Hearing students and
students who were gay, lesbian, bisexual, or other sexual orientation
with psychological abuse and physical abuse. Racial and ethnic
minorities and gay, lesbian, bisexual, and other sexual orientation
students were significantly more at risk for sexual abuse. Gay, lesbian,
bisexual, and other sexual orientation students, students who were
members of a racial or ethnic minority, and female students were
significantly more likely to be raped. Female heterosexual students were
more likely to be the victim of an attempted rape. Suggestions for
further research and policy implications are provided.''

\section{\texorpdfstring{\textcolor[HTML]{5b0057}{gillum2012there}}{}}\label{section-48}

\textsc{\large{'There's So Much at Stake': Sexual Minority Youth Discuss Dating Violence}}
(\emph{Violence Against Women})

``The purpose of this study was to explore perceptions of dating
violence among a sample of sexual minority youth. Focus groups were
conducted as part of a larger study that surveyed 109 sexual minority
youth between the ages of 18 and 24 years. Participants identified four
main themes contributing to dating violence among same-sex couples:
homophobia (societal and internalized); negotiating socially prescribed
gender roles; assumed female connection; and other relationship issues.
Such information is essential for determining the need for and content
of dating violence services, including education, safety planning, and
referrals for mental and physical health services for sexual minority
youth.''

\section{\texorpdfstring{\textcolor[HTML]{5b0057}{goldberg2013sexual}}{}}\label{section-49}

\textsc{\large{Sexual Orientation Disparities in History of Intimate Partner Violence: Results From the California Health Interview Survey}}
(\emph{Journal of Interpersonal Violence})

``Few studies have examined history of intimate partner violence (IPV)
among sexual minorities. We assessed prevalence and predictors of IPV
using a probability sample of California residents ages 18 to 70.
Lifetime and 1-year IPV prevalence was higher in sexual minorities
compared with heterosexuals but this was significant only for bisexual
women and gay men. IPV of bisexual women, but not gay men, occurred in a
heterosexual relationship. We tested whether the higher prevalence of
IPV in gay men and bisexual women was explained by two mental health
indicators-psychological distress and binge drinking-but this hypothesis
was not supported.''

\section{\texorpdfstring{\textcolor[HTML]{5b0057}{kanuha2013relationships}}{}}\label{section-50}

\textsc{\large{'Relationships so loving and so hurtful': The constructed duality of sexual and racial/ethnic intimacy in the context of violence in Asian and Pacific Islander lesbian and queer women's relationships}}
(\emph{Violence Against Women})

``Intimate partner violence (IPV) in Asian, Pacific Islander, and Native
Hawaiian (APINH) queer women's and lesbian relationships was examined
through interviews with 24 APINH respondents. Seven major themes were
uncovered in the dynamics of intimate violence: (a) control,
intimidation, and instilling fear; (b)''deep" emotional intimacy; (c)
first, early, or rebound relationships; (d) sexual jealousy and
possessiveness; (e) shame as a barrier; (f) limited social and potential
partner networks; and (g) crossing/intersecting gender in the ``butch''
as victim. Study implications include expanding research on same-sex IPV
focusing on the intersection of ethnicity, gender, and sexual identity."

\section{\texorpdfstring{\textcolor[HTML]{5b0057}{witte2013social}}{}}\label{section-51}

\textsc{\large{Social Norms for Intimate Partner Violence}}
(\emph{Violence and Victims})

``This study investigated perceived descriptive norms (i.e., perceived
prevalence) for intimate partner violence (IPV) among college students.
Male and female college students were asked to estimate the prevalence
of IPV for same-sex''typical students" on their campus. Perpetrators of
IPV made higher estimates than nonperpetrators. Both perpetrators and
nonperpetrators overestimated the prevalence of IPV when compared to
actual prevalence rates. Findings lend support for using
social-norms-based prevention programs on college campuses."

\section{\texorpdfstring{\textcolor[HTML]{5b0057}{finneran2014antecedents}}{}}\label{section-52}

\textsc{\large{Antecedents of Intimate Partner Violence Among Gay and Bisexual Men}}
(\emph{Violence and Victims})

``Examinations of gay and bisexual men's (GBM) perceptions of intimate
partner violence (IPV), including their perceptions of events likely to
precipitate IPV, are lacking. Focus group discussions with GBM (n = 83)
yielded 24 unique antecedents, or triggers, of IPV in male-male
relationships. Venue-recruited survey participants (n = 700) identified
antecedents that were likely to cause partner violence in male-male
relationships, including antecedents GBM-specific currently absent from
the literature. Chi-square tests found significant variations in
antecedent endorsement when tested against recent receipt of IPV. Linear
regression confirmed that men reporting recent IPV endorsed
significantly more IPV antecedents than men without recent IPV (beta =
1.8155, p \textless{} .012). A better understanding of the IPV event
itself in male-male couples versus heterosexual couples, including its
antecedents, can inform and strengthen IPV prevention efforts.''

\section{\texorpdfstring{\textcolor[HTML]{5b0057}{lewis2014sexual}}{}}\label{section-53}

\textsc{\large{Sexual Minority Stressors and Psychological Aggression in Lesbian Women's Intimate Relationships: The Mediating Roles of Rumination and Relationship Satisfaction}}
(\emph{Psychology of Women Quarterly})

``Our study examined how two sexual minority stressors (internalized
homophobia and social constraints in talking with others about one's
minority sexual identity) are related to psychological aggression (PA)
in lesbian women's relationships. PA includes a range of methods to
hurt, coerce, control, and intimidate intimate partners. Rumination
(i.e., brooding about one's self and life situation) and relationship
satisfaction were examined as potential mediating variables.
Self-identified lesbian women in a same-sex relationship (N = 220) were
recruited from a market research firm's online panel. Participants
completed measures of internalized homophobia, social constraints,
rumination, relationship satisfaction, and frequency of past year PA
victimization and perpetration. Internalized homophobia and social
constraints in talking to friends about sexual identity yielded a
positive indirect link with PA via a sequential path through rumination
and relationship satisfaction. There was an additional indirect positive
association of minority stressors with PA via a unique path through
rumination. These results demonstrate the importance of continued
efforts toward reducing minority stress, where possible, as well as
enhancing coping. Given the importance of rumination and relationship
satisfaction in the link between minority stressors and PA, it is
imperative to improve adaptive coping responses to sexual minority
stressors. Development and validation of individual- and couples-based
interventions that address coping with sexual minority stressors using
methods that decrease rumination and brooding and increase relationship
satisfaction are certainly warranted.''

\section{\texorpdfstring{\textcolor[HTML]{5b0057}{mustanski2014syndemic}}{}}\label{section-54}

\textsc{\large{A Syndemic of Psychosocial Health Disparities and Associations With Risk for Attempting Suicide Among Young Sexual Minority Men}}
(\emph{American Journal of Public Health})

``Objectives. We examined a syndemic of psychosocial health issues among
young men who have sex with men (MSM), with men and women (MSMW), and
with women (MSW). We examined hypothesized drivers of syndemic
production and effects on suicide attempts. Methods. Using a pooled data
set of 2005 and 2007 Youth Risk Behavior Surveys from 11 jurisdictions,
we used structural equation modeling to model a latent syndemic factor
of depression symptoms, substance use, risky sex, and intimate partner
violence. Multigroup models examined relations between victimization and
bullying experiences, syndemic health issues, and serious suicide
attempts. Results. We found experiences of victimization to increase
syndemic burden among all male youths, especially MSMW and MSM compared
with MSW (variance explained = 44\%, 38\%, and 10\%, respectively). The
syndemic factor was shown to increase the odds of reporting a serious
suicide attempt, particularly for MSM (odds ratio \{{[}\}OR{]} = 5.75;
95\% confidence interval \{{[}\}CI{]} = 1.36, 24.39; P \textless{} .001)
and MSMW (OR = 5.08; 95\% CI = 2.14, 12.28; P \textless{} .001) compared
with MSW (OR = 3.47; 95\% CI = 2.50, 4.83; P \textless{} .001).
Conclusions. Interventions addressing multiple psychosocial health
outcomes should be developed and tested to better meet the needs of
young MSM and MSMW.''

\section{\texorpdfstring{\textcolor[HTML]{5b0057}{tran2014prevalence}}{}}\label{section-55}

\textsc{\large{Prevalence of Substance Use and Intimate Partner Violence in a Sample of A/PI MSM}}
(\emph{Journal of Interpersonal Violence})

``This study evaluates the prevalence of three forms of intimate partner
violence (IPV) (i.e., experience of physical, psychological/symbolic,
and sexual battering) among a national sample of Asian/Pacific Islander
(A/PI) men who have sex with men (MSM) in the United States and
identifies their characteristics. The study also reports the differences
of substance use behavior between MSM with and without a previous
history of IPV. Our sample was recruited through venue-based sampling
from seven metropolitan cities as part of the national Men of Asia
Testing for HIV (MATH) study. Among 412 MSM, 29.1\% experienced IPV
perpetrated from a boyfriend or same-gender partner in the past 5 years.
Within the previous 5 years, 62.5\%, 78.3\%, and 40.8\% of participants
experienced physical, psychological/symbolic, and sexual battering,
respectively. Collectively, 35.8\% of participants reported that they
have experienced at least one type of victimization and 64.2\% have
experienced multiple victimizations (two or three types of battering
victimization). Overall, 21.2\% of our sample reported any substance use
within the past 12 months. The present findings suggest that individuals
with a history of IPV in the past 5 years were more likely to report
substance use (33.6\%) compared to those without a history of IPV
experience (16.1\%).''

\section{\texorpdfstring{\textcolor[HTML]{5b0057}{edwards2015physical}}{}}\label{section-56}

\textsc{\large{Physical Dating Violence, Sexual Violence, and Unwanted Pursuit Victimization: A Comparison of Incidence Rates Among Sexual-Minority and Heterosexual College Students}}
(\emph{Journal of Interpersonal Violence})

``The purpose of this study was to estimate the 6-month incidence rates
of sexual assault, physical dating violence (DV), and unwanted pursuit
(e.g., stalking) victimization among sexual-minority (i.e., individuals
with any same-sex sexual experiences) college students with comparison
data from non-sexual-minority (i.e., individuals with only heterosexual
sexual experiences) college students. Participants (N = 6,030) were
primarily Caucasian (92.7\%) and non-sexual-minority (82.3\%). Compared
with nonsexual- minority students (N-SMS; n = 4,961), sexual-minority
students (SMS; n = 1,069) reported significantly higher 6-month
incidence rates of physical DV (SMS: 30.3\%; N-SMS: 18.5\%), sexual
assault (SMS: 24.3\%; N-SMS: 11.0\%), and unwanted pursuit (SMS: 53.1\%;
N-SMS: 36.0\%) victimization. We also explored the moderating role of
gender and found that female SMS reported significantly higher rates of
physical DV than female N-SMS, whereas male SMS and male N-SMS reported
similar rates of physical DV. Gender did not moderate the relationship
between sexual-minority status and victimization experiences for either
unwanted pursuit or sexual victimization. These findings underscore the
alarmingly high rates of interpersonal victimization among SMS and the
critical need for research to better understand the explanatory factors
that place SMS at increased risk for interpersonal victimization.''

\section{\texorpdfstring{\textcolor[HTML]{5b0057}{kubicek2015same-sex}}{}}\label{section-57}

\textsc{\large{'Same-Sex Relationship in a Straight World': Individual and Societal Influences on Power and Control in Young Men's Relationships}}
(\emph{Journal of Interpersonal Violence})

``Young men who have sex with men (YMSM) continue to experience higher
rates of HIV infection than other populations. Recently, there have been
recommendations to consider HIV prevention at the dyadic or couple
level. Using a dyadic approach to HIV prevention would also address an
unaddressed concern related to intimate partner violence (IPV) among
YMSM. Although research on IPV among YMSM is still in its infancy, great
strides have been made in the past 10 years to describe the prevalence
and related correlates of IPV within older adult same-sex relationships.
These studies have found rates of IPV among MSM to be similar to rates
among heterosexual women, and to be on the rise. The present study is
designed to provide insight into how power is conceptualized within YMSM
relationships and the role it may play in relationship challenges. This
study draws from qualitative data collected from 11 focus groups with 86
YMSM and 26 individual semi-structured interviews to understand
relationship challenges and the experiences of YMSM involved in partner
violence. YMSM described relationship power as stemming from numerous
sources including sexual positioning, gender roles, education, income,
prior relationship experiences, and internalized homophobia. The
findings have a number of implications for service providers and program
design. Interventionists and other researchers need to consider power
dynamics and other contextual elements of IPV before effective
interventions can be developed for YMSM and other sexual minority
populations.''

\section{\texorpdfstring{\textcolor[HTML]{5b0057}{lewis2015emotional}}{}}\label{section-58}

\textsc{\large{Emotional distress, alcohol use, and bidirectional partner violence among lesbian women}}
(\emph{Violence Against Women})

``This study examined the relationship between emotional distress
(defined as depression, brooding, and negative affect), alcohol
outcomes, and bidirectional intimate partner violence among lesbian
women. Results lend support to the self-medication hypothesis, which
predicts that lesbian women who experience more emotional distress are
more likely to drink to cope, and in turn report more alcohol use,
problem drinking, and alcohol-related problems. These alcohol outcomes
were, in turn, associated with bidirectional partner violence (BPV).
These results offer preliminary evidence that, similar to findings for
heterosexual women, emotional distress, alcohol use, and particularly,
alcohol-related problems are risk factors for BPV among lesbian women.''

\section{\texorpdfstring{\textcolor[HTML]{5b0057}{sylaska2015disclosure}}{}}\label{section-59}

\textsc{\large{Disclosure Experiences of Sexual Minority College Student Victims of Intimate Partner Violence}}
(\emph{American Journal of Community Psychology})

``Although research on disclosure following intimate partner violence
(IPV) victimization is burgeoning, sexual minority young adults'
(lesbian, gay, bisexual, queer, questioning, etc.; LGBQ+) experiences
have not received equal attention. The current study employed the
minority stress framework to examine disclosure experiences of LGBQ+
college students across the United States reporting physical IPV
victimization within their current relationship (n = 77). Participants
completed measures assessing minority stress and IPV disclosure, and
answered open-ended questions regarding the most and least helpful
persons/responses to disclosure or reasons for non-disclosure. Results
indicated that approximately one-third (35 \%) of victims disclosed to
at least one person, with friends being the most common recipients.
Thematic analyses indicated that talking or listening to the victim was
considered the most helpful response and not understanding the situation
least helpful. Reasons for non-disclosure centered on themes of the
victims' perception that the IPV was not a big deal. Quantitative
findings regarding physical IPV disclosure indicated that non-disclosers
experienced greater minority stress than disclosers. The current study
suggests the presence of differences between sexual minority (i.e., LGBQ
+persons) and non-sexual minority persons, as well as between LGBQ+
young adults/college students and older adults and presents a
theoretical structure (i.e., minority stress framework) through which
these differences may be understood.''

\section{\texorpdfstring{\textcolor[HTML]{5b0057}{witte2015perceived}}{}}\label{section-60}

\textsc{\large{Perceived Social Norms for Intimate Partner Violence in Proximal and Distal Groups}}
(\emph{Violence and Victims})

``This study investigated students' perceived descriptive social norms
for intimate partner violence (IPV) among proximal and distal groups at
college. Male and female college students estimated the prevalence rates
for IPV among same-sex friends (proximal group) and same-sex''typical
students" (distal group). In separate regression equations for men and
women, perceived estimates of IPV rates for same-sex friends, but not
estimates for same-sex typical students, were positively related with
the participants' own IPV behaviors. Findings have important
implications for IPV prevention and intervention programs for college
students."

\section{\texorpdfstring{\textcolor[HTML]{5b0057}{wu2015association}}{}}\label{section-61}

\textsc{\large{The Association Between Substance Use and Intimate Partner Violence Within Black Male Same-Sex Relationships}}
(\emph{Journal of Interpersonal Violence})

``Compared with the extant research on heterosexual intimate partner
violence (IPV)-including the knowledge base on alcohol and illicit drug
use as predictors of such IPV-there is a paucity of studies on IPV among
men who have sex with men (MSM), especially Black MSM. This study
investigates the prevalence of experiencing and perpetrating IPV among a
sample of Black MSM couples and examines whether heavy drinking and/or
illicit substance use is associated with IPV. We conducted a secondary
analysis on a data set from 74 individuals (constituting 37 Black MSM
couples) screened for inclusion in a couple-based HIV prevention pilot
study targeting methamphetamine-involved couples. More than one third
(n=28, 38\%) reported IPV at some point with the current partner: 24
both experiencing and perpetrating, 2 experiencing only, and 2
perpetrating only. IPV in the past 30 days was reported by 21 (28\%) of
the participants: 18 both experiencing and perpetrating, 1 experiencing
only, and 2 perpetrating only. Heavy drinking and methamphetamine use
each was associated significantly with experiencing and perpetrating IPV
throughout the relationship as well as in the past 30 days. Rock/crack
cocaine use was significantly associated with any history of
experiencing and perpetrating IPV. Altogether, IPV rates in this sample
of Black MSM couples equal or exceed those observed among women
victimized by male partners as well as the general population of MSM.
This exploratory study points to a critical need for further efforts to
understand and address IPV among Black MSM. Similar to heterosexual IPV,
results point to alcohol and illicit drug use treatment as important
avenues to improve the health and social well-being of Black MSM.''

\section{\texorpdfstring{\textcolor[HTML]{5b0057}{dixon2016association}}{}}\label{section-62}

\textsc{\large{The Association of Investment Model Variables and Dyadic Patterns of Physical Partner Violence: A Study of College Women}}
(\emph{Journal of Interpersonal Violence})

``Previous research has examined the association between intimate
partner violence (IPV) victimization experiences and investment model
variables, particularly with relation to leaving intentions. However,
research only has begun to explore the impact that various dyadic
patterns of IPV (i.e., unidirectional victimization, unidirectional
perpetration, bidirectional violence, and non-violence) have on
investment model variables. Grounded in behavioral principles, the
current study used a sample of college women to assess the impact that
perpetration and victimization have on investment model variables.
Results indicated that 69.2\% of the sample was in a relationship with
no IPV. Among those who reported IPV in their relationships, 11.9\%
reported unidirectional perpetration, 10.6\% bidirectional violence, and
7.4\% unidirectional victimization. Overall, the findings suggest that
women's victimization (i.e., victim only and bidirectional IPV) is
associated with lower levels of satisfaction and commitment, and that
women's perpetration (i.e., perpetration only and bidirectional IPV) is
associated with higher levels of investment. Women in bidirectionally
violent relationships reported higher quality alternatives than women in
non-violent relationships. The current study emphasizes the importance
of considering both IPV perpetration and IPV victimization experiences
when exploring women's decisions to remain in relationships.''

\section{\texorpdfstring{\textcolor[HTML]{5b0057}{edwards2016college}}{}}\label{section-63}

\textsc{\large{College Campus Community Readiness to Address Intimate Partner Violence Among LGBTQ plus Young Adults: A Conceptual and Empirical Examination}}
(\emph{American Journal of Community Psychology})

``This paper provides an overview of a conceptual model that integrates
theories of social ecology, minority stress, and community readiness to
better understand risk for and outcomes of intimate partner violence
(IPV) among LGBTQ+ college students. Additionally, online survey data
was collected from a sample of 202 LGBTQ+ students enrolled in 119
colleges across the United States to provide preliminary data on some
aspects of the proposed model. Results suggested that students generally
thought their campuses were low in readiness to address IPV; that is,
students felt that their campuses could do more to address IPV and
provide IPV services specific to LGBTQ+ college students. Perceptions of
greater campus readiness to address IPV among LGBTQ+ college students
was significantly and positively related to a more favorable LGBTQ+
campus climate and a greater sense of campus community. Additionally,
IPV victims were more likely to perceive higher levels of campus
community readiness than non-IPV victims. There was no association
between IPV perpetration and perceptions of campus community readiness.
Greater sense of community was marginally and inversely related to IPV
victimization and perpetration. Sense of community and LGBTQ+ campus
climate also varied to some extent as a function of region of the
country and type of institution. Implications for further development
and refinement of the conceptual model, as well as future research
applying this model to better understand IPV among sexual minority
students are discussed.''

\section{\texorpdfstring{\textcolor[HTML]{5b0057}{langenderfer-magruder2016experiences}}{}}\label{section-64}

\textsc{\large{Experiences of Intimate Partner Violence and Subsequent Police Reporting Among Lesbian, Gay, Bisexual, Transgender, and Queer Adults in Colorado: Comparing Rates of Cisgender and Transgender Victimization}}
(\emph{Journal of Interpersonal Violence})

``Research indicates that lesbian, gay, bisexual, transgender, and queer
(LGBTQ) individuals are at high risk of victimization by others and that
transgender individuals may be at even higher risk than their cisgender
LGBQ peers. In examining partner violence in particular, extant
literature suggests that LGBTQ individuals are at equal or higher risk
of partner violence victimization compared with their heterosexual
peers. As opposed to sexual orientation, there is little research on
gender identity and partner violence within the LGBTQ literature. In the
current study, the authors investigated intimate partner violence (IPV)
in a large sample of LGBTQ adults (N = 1,139) to determine lifetime
prevalence and police reporting in both cisgender and transgender
individuals. Results show that more than one fifth of all participants
ever experienced partner violence, with transgender participants
demonstrating significantly higher rates than their cisgender peers.
Implications focus on the use of inclusive language as well as future
research and practice with LGBTQ IPV victims.''



\end{document}
