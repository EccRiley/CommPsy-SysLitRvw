\documentclass[]{tufte-handout}

% ams
\usepackage{amssymb,amsmath}

\usepackage{ifxetex,ifluatex}
\usepackage{fixltx2e} % provides \textsubscript
\ifnum 0\ifxetex 1\fi\ifluatex 1\fi=0 % if pdftex
  \usepackage[T1]{fontenc}
  \usepackage[utf8]{inputenc}
\else % if luatex or xelatex
  \makeatletter
  \@ifpackageloaded{fontspec}{}{\usepackage{fontspec}}
  \makeatother
  \defaultfontfeatures{Ligatures=TeX,Scale=MatchLowercase}
  \makeatletter
  \@ifpackageloaded{soul}{
     \renewcommand\allcapsspacing[1]{{\addfontfeature{LetterSpace=15}#1}}
     \renewcommand\smallcapsspacing[1]{{\addfontfeature{LetterSpace=10}#1}}
   }{}
  \makeatother
\fi

% graphix
\usepackage{graphicx}
\setkeys{Gin}{width=\linewidth,totalheight=\textheight,keepaspectratio}

% booktabs
\usepackage{booktabs}

% url
\usepackage{url}

% hyperref
\usepackage{hyperref}

% units.
\usepackage{units}


\setcounter{secnumdepth}{-1}

% citations

% pandoc syntax highlighting

% longtable

% multiplecol
\usepackage{multicol}

% strikeout
\usepackage[normalem]{ulem}

% morefloats
\usepackage{morefloats}


% tightlist macro required by pandoc >= 1.14
\providecommand{\tightlist}{%
  \setlength{\itemsep}{0pt}\setlength{\parskip}{0pt}}

% title / author / date
\title{Community Psychology and Related Journals}
\author{Rachel M. Smith (via the Society for Community Research \& Action)}
\date{22 March 2017}

\bibliographystyle{tufte}
% \usepackage{caption}
\usepackage{cleveref}
%
% --------------------- %
% Latex Logo Commands
% --------------------- %
%
\usepackage{xspace}
\newcommand{\latex}{\LaTeX\xspace}
\newcommand{\tex}{\TeX\xspace}
\newcommand{\bibtex}{\textsc{Bib}\tex}
%
% --------------------- %
% Colors
% --------------------- %
%
\usepackage{color}
\definecolor{magenta}{rgb}{0.79, 0.08, 0.48} %% ~c9147a %%
\definecolor{dkmagenta}{rgb}{0.55, 0.0, 0.55} %% ~8c008c %%
\definecolor{dpmagenta}{rgb}{0.8, 0.0, 0.8} %% ~140014 %%
\definecolor{patriarch}{rgb}{0.5, 0.0, 0.5} %% ~0d000d %%
\definecolor{dkpatriarch}{rgb}{0.4, 0.0, 0.4} %% ~0a000a %%
\definecolor{blue}{rgb}{0.07, 0.04, 0.56} %% ~120a8f %%
\definecolor{royalblue}{rgb}{0.0, 0.22, 0.66} %% ~0038a8 %%
\definecolor{dkblue}{rgb}{0.0, 0.0, 0.55} %% ~00008c %%
\definecolor{mnblue}{rgb}{0.1, 0.1, 0.44} %% ~030370 %%
\definecolor{smblue}{rgb}{0.0, 0.2, 0.6} %% ~00050f %%
\definecolor{Rblue}{rgb}{0.39, 0.35, 0.639} %% ~0a09a3 %%
\definecolor{dkmnblue}{rgb}{0.0, 0.2, 0.4} %% ~00050a %%
\definecolor{navy}{rgb}{0.0, 0.0, 0.5} %% ~00000d %%
\definecolor{dknavy}{rgb}{0, 0, .208} %% ~000035 %%
\definecolor{blublk}{rgb}{0, 0, .106} %% ~00001b %%
\definecolor{blugray}{rgb}{0.33, 0.41, 0.47} %% ~546978 %%
\definecolor{grayblue}{rgb}{0.33, 0.41, 0.58} %% ~~546994 %%
\definecolor{slgray}{rgb}{0.44, 0.5, 0.56} %% ~70808d %%
\definecolor{red}{rgb}{.545, 0.0, 0.0} %% ~8b0000 %%
\definecolor{dkred}{rgb}{.247, 0.0, 0.0} %% ~3f0000 %%
\definecolor{mplblu}{HTML}{363283}
%
\definecolor{pdxgray}{HTML}{373737} %% ~373737 %%
\definecolor{pdxgreen}{HTML}{8B9535} %% ~8B9535 %%
\definecolor{myblack}{HTML}{181C20} %% ~181C20 %%
%
% ---------------------------------------- %
% Indent first line of text in tabular env %
% ---------------------------------------- %
%
\newcommand{\rowgroup}[2][-1em]{\hspace{#1}#2}
\newcommand{\mrowgroup}[3]{\hspace*{#1}#2\hspace*{#1}#3}
%
% --------------------- %
% Format Block Quotes
% --------------------- %
%   Size: Scriptsize
%   Reduce vertical space above
%   Color: Gray
%
\usepackage{setspace}
% \expandafter\def\expandafter\quote\expandafter{\quote\small\singlespacing\color{myblack!65}\vspace{-0.5\baselineskip}}

% \expandafter\def\expandafter\quote\expandafter{\quote\small\singlespacing\vspace{-1em}}

% \setlength\listindent{1em}

\usepackage{enumitem}
\setlist[itemize, 1]{leftmargin=!, labelindent=0.5em, itemindent=-3em, label=\scriptsize{$\cdot$}, partopsep=0em, topsep=0.15em}
% \setlist[itemize, 2]{leftmargin=4em, label=$\centerdot$, topsep=0em}
% \setlength{\itemindent}{5in}

%
% ------------------- %
% Make Links Standout
% ------------------- %
%   (E.Tufte does not believe in using colors in links. I disagree.) %
%
% \newcommand{\rurl}[1]{\underline{\color{dkblue}{\url{~1}}}}
% \newcommand{\rhref}[2]{\underline{\color{dkblue}{\href{~1}{~2}}}}
\hypersetup{breaklinks=true,colorlinks=true,linkcolor=navy,urlcolor=navy}
%
% ------------------- %
% Format "texttt"
% ------------------- %
%
\newcommand{\rtt}[1]{\color{patriarch}{\texttt{#1}}}
%
\usepackage{amsmath}
%
% ----------------------------- %
% Command to insert "ToDo" tags
% ----------------------------- %
%
\newcommand{\todo}{\textbf{\color{red}{\texttt{[ToDo]}}}}
\newcommand{\inprogress}{\textbf{\textit{\color{blue}{\texttt{[In Progress]}}}}}
\newcommand{\complete}{\sout{\textit{\color{slgray}{\texttt{[Complete]}}}}}

\usepackage{enumitem,amssymb}
\newlist{todolist}{itemize}{2}
\setlist[todolist]{label=$\square$}
\newcommand{\todoitem}[1]{\textit{\color{red}{#1}}}

\newcommand{\textbft}[1]{\underline{\textbf{\texttt{#1}}}}


% ---------------------------%
% Code Formatting %
% ---------------------------%
% \usepackage{highlight}

% \definecolor{fgcolor}{rgb}{0.196, 0.196, 0.196}
% \newcommand{\hlnum}[1]{\textcolor[rgb]{0.063,0.58,0.627}{#1}}%
% \newcommand{\hlstr}[1]{\textcolor[rgb]{0.063,0.58,0.627}{#1}}%
% \newcommand{\hlcom}[1]{\textcolor[rgb]{0.588,0.588,0.588}{#1}}%
% \newcommand{\hlopt}[1]{\textcolor[rgb]{0.196,0.196,0.196}{#1}}%
% \newcommand{\hlstd}[1]{\textcolor[rgb]{0.196,0.196,0.196}{#1}}%
% \newcommand{\hlkwa}[1]{\textcolor[rgb]{0.231,0.416,0.784}{#1}}%
% \newcommand{\hlkwb}[1]{\textcolor[rgb]{0.627,0,0.314}{#1}}%
% \newcommand{\hlkwc}[1]{\textcolor[rgb]{0,0.631,0.314}{#1}}%
% \newcommand{\hlkwd}[1]{\textcolor[rgb]{0.78,0.227,0.412}{#1}}%

\let\hlipl\hlkwb
% \newcommand{\Rrule}{\textcolor{Rblue}{\rule{\linewidth}{0.05mm}}\newline\includegraphics[width=0.5cm]{auxDocs/Rlogo.png}}
\usepackage{dashrule}

\newcommand{\Rrule}{
    % \setlength{\parindent}{-10pt}
    \vspace*{1em}
    \noindent
    \hspace{-1em}
    \includegraphics[width=0.5cm]{auxDocs/Rlogo.png}
    \textcolor{Rblue}{
        \rule[0.1in]{0.90\textwidth}{0.02mm}
    }
    \vspace{-1.35em}
}

\newcommand{\Rerule}{
    % \setlength{\parindent}{-0.5in}
    \noindent
    \hspace{-1em}
    \textcolor{Rblue}{
        $\llcorner$\rule[-0.4mm]{\textwidth}{0.02mm}
                % \hfill
                % $\lrcorner$
    }
}

\newcommand{\Rruleb}{
    % \setlength{\parindent}{-10pt}
    \vspace*{1em}
    \noindent
    \hspace{-1em}
    \includegraphics[width=0.5cm]{auxDocs/Rlogo-bw.png}
    \textcolor{slgray}{
        \rule[0.1in]{0.90\textwidth}{0.02mm}
    }
    \vspace{-1.35em}
}

\newcommand{\Reruleb}{
    % \setlength{\parindent}{-0.5in}
    \noindent
    \hspace{-1em}
    \textcolor{slgray}{
        $\llcorner$\rule[-0.4mm]{\textwidth}{0.02mm}
                % \hfill
                % $\lrcorner$
    }
}

\newcommand{\MPrule}{
    % \setlength{\parindent}{-10pt}
    \vspace*{1em}
    \noindent
    \hspace{-1em}
    \includegraphics[width=0.5cm]{../auxDocs/mplus.png}
    \textcolor{mplblu}{
        \rule[0.1in]{0.90\textwidth}{0.02mm}
    }
    \vspace{-1.5em}
}

\newcommand{\MPerule}{
    % \setlength{\parindent}{-0.5in}
    \noindent
    \hspace{-1em}
    \textcolor{mplblu}{
        $\llcorner$\rule[-0.4mm]{\textwidth}{0.02mm}
                % \hfill
                % $\lrcorner$
    }
}

\newcommand{\Frule}{
    \vspace*{-1em}
    \begin{fullwidth}\textcolor{blublk}{\rule{\linewidth}{0.2mm}}\end{fullwidth}
}

% \newcommand{\Rrule}{
%     \noindent
%     \textcolor{Rblue}{
%         $\ulcorner\textregistered$\hdashrule[0.015in]{
%             0.9\linewidth
%             }
%             {1pt}{1pt}
%         $\textregistered\urcorner$
%         }}
% \newcommand{\Rerule}{
%     \noindent
%     \textcolor{Rblue}{
%         $\llcorner\textregistered$\hdashrule[0.025in]{
%             0.9\linewidth
%             }
%             {1pt}{1pt}
%         $\textregistered\lrcorner$
%         }}
% \newcommand{\Rerule}{\noindent\textcolor{Rblue}{\vdash\hdashrule[-0.015in]{\linewidth}{1pt}{1pt}\dashv}

% \vdash - \dashv
% \perp
% \ll - \gg
% +
% \pm
% \mp
% \ \newcommand{\Rrule}{\noindent\includegraphics[width=0.5cm]{auxDocs/Rlogo.png}\textcolor{Rblue}{\hdashrule[0.25in]{\linewidth}{1pt}{1pt}}}
% \newcommand{\Rrule}{\noindent\includegraphics[width=0.5cm]{auxDocs/Rlogo.png}\textcolor{Rblue}{\rule[0.25in]{\linewidth}{0.05mm}}}

% \newcommand{\Rerule}{\textcolor{Rblue}{\rule{\linewidth}{0.05mm}}}
% \hdashrule [⟨raise⟩] [⟨leader⟩] {⟨width⟩} {⟨height⟩} {⟨dash⟩}

%%%%%%%%%%%%%%%%%%%%%%%%%%
% Wrapper for Chi-Square %
%%%%%%%%%%%%%%%%%%%%%%%%%%

\newcommand{\chisq}{\chi^{2}}
\newcommand{\sq}{^{2}}

\newcommand{\tdef}[3][-0.5em]{\tufteskip\noindent\rowgroup[#1]{\textsc{#2}} \newline {#3}}
\newcommand{\hf}{\hfill}
\DeclareMathAlphabet{\mathpzc}{OT1}{pzc}{m}{it} %% to make \mathpzc typeset its argument in Zapf Chancery (see page 16 of "The Great, Big List of LATEX Symbols" by David Carlisle, Scott Pakin, & Alexander Holt (2001)) %%

\newcommand{\df}{\mathpzc{df}}

\begin{document}

\maketitle




Below is a list of academic journals specific or related to Community
Psychology research and practice compiled by the
\href{http://www.scra27.org}{\emph{Society for Community Research \&
Action (SCRA)}}.\\
\marginnote{Read more at http://www.scra27.org/publications/other-journals-relevant-community-psychology/}

\begin{enumerate}
\def\labelenumi{\arabic{enumi}.}
\tightlist
\item
  \emph{Action Research}
\item
  \emph{American Journal of Community Psychology}
\item
  \emph{American Journal of Health Promotion}
\item
  \emph{American Journal of Orthopsychiatry}
\item
  \emph{American Journal of Preventive
  Medicine\textsuperscript{\textdagger}}
\item
  \emph{American Journal of Public Health}
\item
  \emph{Australian Community Psychologist\textsuperscript{\textdagger}}
\item
  \emph{Community Development\textsuperscript{\textdagger}}
\item
  \emph{Community Development Journal}
\item
  \emph{Community Mental Health Journal}
\item
  \emph{Community Psychology in Global
  Perspective\textsuperscript{\textdagger}}
\item
  \emph{Cultural Diversity \& Ethnic Minority
  Psychology\textsuperscript{\textdagger}}
\item
  \emph{Global Journal of Community Psychology
  Practice\textsuperscript{\textdagger}}
\item
  \emph{Health Education \& Behavior}
\item
  \emph{Health Promotion Practice}
\item
  \emph{Journal of Applied Social Psychology}
\item
  \emph{Journal of Community \& Applied Social Psychology}
\item
  \emph{Journal of Community Practice}
\item
  \emph{Journal of Health \& Social Behavior}
\item
  \emph{Journal of Prevention \& Intervention}
\item
  \emph{Journal of Primary Prevention}
\item
  \emph{Journal of Rural Community
  Psychology\textsuperscript{\textdagger}}
\item
  \emph{Journal of Social Issues}
\item
  \emph{Journal of Community Psychology}
\item
  \emph{Psychiatric Rehabilitation Journal}
\item
  \emph{Psychology of Women Quarterly}
\item
  \emph{Psychosocial Intervention}
\item
  \emph{Social Science \& Medicine}
\item
  \emph{The Community Psychologist\textsuperscript{\textdaggerdbl}}
\item
  \emph{Transcultural Psychiatry}
\item
  \emph{Progress in Community Health Partnerships: Research, Education,
  \& Action}
\end{enumerate}

\marginnote{\textsc{Note:} \textdagger Indicates journals for which no \textit{\href{http://www.scimagojr.com}{Scimago Journal Rank (SJR)}} impact is available. \textdaggerdbl Indicates informal publications. }

\begin{center}\rule{0.5\linewidth}{\linethickness}\end{center}

\begin{description}
\tightlist
\item[\href{http://arj.sagepub.com/}{\textsc{Action Research}}.]
\emph{Action Research} is an international, interdisciplinary, peer
reviewed, quarterly published refereed journal which is a forum for the
development of the theory and practice of action research. The journal
publishes quality articles on accounts of action research projects,
explorations in the philosophy and methodology of action research, and
considerations of the nature of quality in action research practice.
This journal is a member of the
\href{http://publicationethics.org/about}{Committee on Publication
Ethics (COPE)}
\item[\href{http://www.scra27.org/publications/ajcp/}{\textsc{American Journal of Community Psychology}}.]
\emph{American Journal of Community Psychology (AJCP)} publishes
original quantitative, qualitative, and mixed methods research;
theoretical papers; empirical reviews; reports of innovative community
programs or policies; and first person accounts of stakeholders involved
in research, programs, or policy.
\item[\href{https://www.healthpromotionjournal.com/}{\textsc{American Journal of Health Promotion}}.]
The \emph{American Journal of Health Promotion} is a peer-reviewed
journal on the science of lifestyle change. The editorial goal of the
American Journal of Health Promotion is to provide a forum for exchange
among the many disciplines involved in health promotion and an interface
between researchers and practitioners.
\item[\href{http://www.apa.org/pubs/journals/ort/}{\textsc{American Journal of Orthopsychiatry}}.]
The \emph{American Journal of Orthopsychiatry (AJO)} reflects the
Association's purpose: ``to facilitate the generation and exchange of
knowledge relevant to the development and implementation of policies and
practices consistent with the promotion of mental health and social
justice, including the protection of human rights.'' Consistent with
that mission, the journal publishes articles that clarify, challenge, or
reshape the prevailing understanding of factors in the prevention and
correction of injustice and in the sustainable development of a humane
and just society. AJO publishes theoretical, policy--analytic, and
empirical articles on topics related to the Association's historic
values and themes.
\item[\href{http://www.ajpmonline.org/home}{\textsc{American Journal of Preventive Medicine }}.]
The \emph{American Journal of Preventive Medicine} is the official
journal of the American College of Preventive Medicine and the
Association for Prevention Teaching and Research. It publishes articles
in the areas of prevention research, teaching, practice and policy.
Original research is published on interventions aimed at the prevention
of chronic and acute disease and the promotion of individual and
community health. Of particular emphasis are papers that address the
primary and secondary prevention of important clinical, behavioral and
public health issues such as injury and violence, infectious disease,
women's health, smoking, sedentary behaviors and physical activity,
nutrition, diabetes, obesity, and alcohol and drug abuse. Papers also
address educational initiatives aimed at improving the ability of health
professionals to provide effective clinical prevention and public health
services. Papers on health services research pertinent to prevention and
public health are also published. The journal also publishes official
policy statements from the two co-sponsoring organizations, review
articles, media reviews, and editorials. Finally, the journal
periodically publishes supplements and special theme issues devoted to
areas of current interest to the prevention community.
\item[\href{http://ajph.aphapublications.org/}{\textsc{American Journal of Public Health}}.]
The \emph{American Journal of Public Health (AJPH)} is dedicated to
publication of original work in research, research methods, and program
evaluation in the field of public health. The Journal also regularly
publishes editorials and commentaries and serves as a forum for health
policy analysis. The mission of the Journal is to advance public health
research, policy, practice, and education. Each month, national and
international public health professionals turn to AJPH for the most
current, authoritative, in-depth information in the field.
\item[\href{http://www.groups.psychology.org.au/ccom/publications/}{\textsc{Australian Community Psychologist}}.]
The \emph{Australian Community Psychologist} publishes work that is of
relevance to community psychology practitioners and researchers and
others interested in the field. Local and national manuscripts may be
published, and submissions by emerging researchers and practitioners are
encouraged. The journal features contributions that are state of the art
reviews of professional and applied areas, and reviews and essays on
matters of general relevance to community psychologists, in addition to
empirical research reports relevant to community psychologists. The
journal also features individual manuscripts and collections of
manuscripts which address matters of general, professional and public
relevance, techniques and approaches in psychological practice,
professional development issues, and professional and public policy
issues, and reviews of books which relate directly to the major areas of
practice in community psychology.
\item[\href{http://www.tandfonline.com/toc/rcod20/current\#.VZgph2Bh1E5}{\textsc{Community Development}}.]
\emph{Community Development} is the peer-reviewed journal of the
Community Development Society. Published five times per year, Community
Development is devoted to improving knowledge and practice in the field
of purposive community change. The mission of the journal is to advance
critical theory, research, and practice in all domains of community
development, including sociocultural, political, environmental, and
economic.
\item[\href{http://cdj.oxfordjournals.org/}{\textsc{Community Development Journal}}.]
The \emph{Community Development Journal} is the leading international
journal in its field, covering a wide range of topics, reviewing
significant developments and providing a forum for cutting-edge debates
about theory and practice. It adopts a broad definition of community
development to include policy, planning and action as they impact on the
life of communities. It seeks to publish critically focused articles
which challenge received wisdom, report and discuss innovative
practices, and relate issues of community development to questions of
social justice, diversity and environmental sustainability.
\item[\href{http://link.springer.com/journal/10597}{\textsc{Community Mental Health Journal}}.]
The \emph{Community Mental Health Journal} is devoted to the evaluation
and improvement of public sector mental health services for people
affected by severe mental disorders, serious emotional disturbances
and/or addictions. Coverage includes: nationally representative
epidemiologic projects intervention research involving benefit and risk
comparisons between service programs methodology, such as
instrumentation, where particularly pertinent to public sector
behavioral health evaluation or research
\item[\href{http://siba-ese.unisalento.it/index.php/cpgp}{\textsc{Community Psychology in Global Perspective}}.]
\emph{Community Psychology in Global Perspective} is an open access,
on-line, interdisciplinary, peer-reviewed journal devoted to research,
theory and intervention, and review articles exploring human
interactions in community settings across the globe. Its special focus
is on making explicit the ways in which culture acts as a framework
organizing and guiding our experiences, utilizing ecological
perspectives to enhance our understanding and promotion of individual
and community well-being, and advancing work aimed at the creation of
positive social change and social justice. It especially encourages
submissions of field-based, culturally situated research and
intervention, as well as psychologically framed qualitative research.
\item[\href{http://www.apa.org/pubs/journals/cdp/}{\textsc{Cultural Diversity \& Ethnic Minority Psychology}}.]
\emph{Cultural Diversity \& Ethnic Minority Psychology} seeks to advance
the psychological science of culture, ethnicity, and race through the
publication of empirical research, as well as theoretical, conceptual,
and integrative review articles that will stimulate further empirical
research, on basic and applied psychological issues relevant to racial
and ethnic groups that have been historically subordinated,
underrepresented, or underserved.
\item[\href{http://www.gjcpp.org/}{\textsc{Global Journal of Community Psychology Practice}}.]
The \emph{Global Journal of Community Psychology Practice (GJCCP)} is an
e-journal for practitioners of community psychology and community
improvement around the globe. We look forward to working with
practitioners and applied researchers to share quality work and to
foster a learning community that will contribute to ongoing advances in
the broad field of Community Practice, both in psychology and related
disciplines. We seek contributions from community practitioners in many
fields, including community psychology, but also including community
development, public health, community organizing and others. Please
consider sharing your knowledge, insights and accomplishments with the
practice community along with innovations that may help communities
throughout the world.
\item[\href{http://heb.sagepub.com/}{\textsc{Health Education \& Behavior}}.]
The \emph{Health Education \& Behavior (HEB)} is a peer-reviewed
bi-monthly journal that provides empirical research, case studies,
program evaluations, literature reviews, and discussions of theories of
health behavior and health status, as well as strategies to improve
social and behavioral health. HEB also examines the processes of
planning, implementing, managing, and assessing health education and
social-behavioral interventions. This journal is a member of the
Committee on Publication Ethics (COPE).
\item[\href{http://hpp.sagepub.com/}{\textsc{Health Promotion Practice}}.]
The \emph{Health Promotion Practice (HPP)} publishes authoritative,
peer-reviewed articles devoted to the practical application of health
promotion and education. The journal is unique in its focus on critical
and strategic information for professionals engaged in the practice of
developing, implementing, and evaluating health promotion and disease
prevention programs. Health Promotion Practice will serve as a forum to
explore the applications of health promotion/public health education
interventions programs and best practice strategies in various settings,
including but not limited to: community, health care, worksite,
educational and international settings. It also examines
practice-related issues, including program descriptions, teaching
methods, needs assessment tools and methodologies, intervention
strategies, health promotion, problem-solving issues, and evaluation
presentations.
\item[\href{http://www.wiley.com/WileyCDA/WileyTitle/productCd-JASP.html}{\textsc{Journal of Applied Social Psychology}}.]
The \emph{Journal of Applied Social Psychology} is a monthly publication
devoted to applications of experimental behavioral science research to
problems of society (e.g., organizational and leadership psychology,
safety, health, and gender issues; perceptions of war and natural
hazards; jury deliberation; performance, AIDS, cancer, heart disease,
exercise, and sports).
\item[\href{http://www.wiley.com/WileyCDA/WileyTitle/productCd-CASP.html}{\textsc{Journal of Community \& Applied Social Psychology}}.]
The \emph{Journal of Community and Applied Social Psychology} aims to
complete a thorough review and assessment of papers, returning a first
decision to authors, within 12 weeks of submission.
\item[\href{https://www.acosa.org/joomla/journalinfo}{\textsc{Journal of Community Practice}}.]
The \emph{Journal of Community Practice: Organizing, Planning,
Development \& Change} is a interdisciplinary journal designed to
provide a forum for community practice, including community organizing,
planning, social administration, organizational development, community
development and social change. The journal contributes to the
development of knowledge related to numerous disciplines, including
social work and the social sciences, urban planning, social and economic
development, community organizing, policy analysis, urban and rural
sociology, public administration, and nonprofit management. The Journal
of Community Practice articulates contemporary issues, providing
direction on how to think about social problems, developing approaches
to dealing with them, and outlining ways to implement these concepts in
classrooms and practice settings. As a forum for authors and a resource
for readers, the Journal of Community Practice makes an invaluable
contribution to community practice its conceptualization, applications,
and practice. As the only journal focusing on community practice, it
covers research, theory, practice, and curriculum strategies for the
full range of work with communities and organizations.
\item[\href{http://onlinelibrary.wiley.com/journal/10.1002/(ISSN)1520-6629}{\textsc{Journal of Community Psychology}}.]
The \emph{Journal of Community Psychology} is a peer-reviewed journal
devoted to research, evaluation, assessment and intervention, and review
articles that deal with human behavior in community settings. Articles
of interest include descriptions and evaluations of service programs and
projects, studies of youth, parenting, and family development,
methodology and design for work in the community, the interaction of
groups in the larger community, and criminals and corrections.
\item[\href{http://hsb.sagepub.com/}{\textsc{Journal of Health \& Social Behavior}}.]
The \emph{Journal of Health and Social Behavior (JHSB)} is a medical
sociology journal that publishes empirical and theoretical articles that
apply sociological concepts and methods to the understanding of health
and illness and the organization of medicine and health care. Its
editorial policy favors manuscripts that are grounded in important
theoretical issues in medical sociology or the sociology of mental
health and that advance theoretical understanding of the processes by
which social factors and human health are inter-related.
\item[\href{http://www.tandfonline.com/action/journalInformation?journalCode=wpic20\#.U1p9ieZdXTM}{\textsc{Journal of Prevention \& Intervention in the Community}}.]
The \emph{Journal of Prevention \& Intervention in the Community} is on
the cutting edge of social action and change, not only covering current
thought and developments, but also defining future directions in the
field. Under the editorship of Joseph R. Ferrari since 1995, Prevention
in Human Services was retitled as theJournal of Prevention \&
Intervention in the Community to reflect its focus of providing
professionals with information on the leading, effective programs for
community intervention and prevention of problems. Because of its
intensive coverage of selected topics and the sheer length of each
issue, the Journal of Prevention \& Intervention in the Community is the
first--and in many cases, primary--source of information for mental
health and human services development. This innovative journal is of
interest not only to human services program administrators, clinical
supervisors, planners, education specialists, and researchers, but also
to health care and helping professionals in other fields where new
methods of services delivery and new models of practice can be achieved
within the community.
\item[\href{http://link.springer.com/journal/10935}{\textsc{Journal of Primary Prevention}}.]
The \emph{Journal of Primary Prevention} is dedicated to a better
understanding of primary prevention theory, practice, and research.
Primary prevention involves preventing predictable and interrelated
problems, protecting existing states of health and healthy functioning,
and promoting psychosocial wellness for identified populations of
people. The journal publishes along the spectrum of prevention research
including research on determinants, intervention development,
intervention evaluation, methodology, dissemination, policy-to-practice,
and meta-analyses. JPP particularly welcomes manuscripts on behaviors
that may be established during childhood and adolescence that can lead
to the major causes of morbidity, mortality, and poor quality of life in
adult years. Both qualitative and quantitative studies are welcome.
Content areas include, but are not limited to, intentional and
unintentional injury, substance use, sexual behaviors, physical
activity, nutrition, mental health, and school functioning. The journal
also focuses on the major causes of disparities, social determinants,
school- and community-based programs, cross-cultural comparisons,
community-based participatory research, and factors contributing to
social injustice.Specific types of papers published in the journal
include original research, research methods and practice, brief reports,
and literature reviews.
\item[\href{http://www.marshall.edu/jrcp/}{\textsc{Journal of Rural Community Psychology}}.]
The \emph{Journal of Rural Community Psychology (JRCP)} is a
peer-reviewed, scholarly journal published in electronic form by
Marshall University, in Huntington, West Virginia. This journal has been
devoted to the dissemination of information related to the sociological,
psychological and mental health issues in rural and small community
settings.
\item[\href{https://www.spssi.org/index.cfm?fuseaction=page.viewpage\&pageid=950}{\textsc{Journal of Social Issues}}.]
The \emph{Journal of Social Issues (JSI)} is the flagship journal of the
Society for the Psychological Study of Social Issues. The goal of JSI is
to communicate scientific findings and interpretations relevant to
pressing social issues in a non-technical manner but without the
sacrifice of professional standards. Each issue of JSI is organized
around an integral theme. Issues of the Journal are proposed and
developed by social researchers, who serve as issue editors under the
direction of the JSI board. JSI does not publish unsolicited manuscripts
or book reviews. Our sister journal, Analyses of Social Issues and
Public Policy (ASAP) accepts unsolicited manuscripts and book reviews.
\item[\href{http://www.uspra.org/knowledge-center/psychiatric-rehabilitation-journal}{\textsc{Psychiatric Rehabilitation Journal}}.]
The \emph{Psychiatric Rehabilitation} Journal is the official
professional journal of PRA and features original contributions related
to the rehabilitation, psychosocial treatment, and recovery of people
with serious mental illnesses. PRJ's target audience includes
psychiatric rehabilitation practitioners and researchers, as well as
recipients of mental health and rehabilitation services.
\item[\href{http://www.apadivisions.org/division-35/publications/journal/index.aspx}{\textsc{Psychology of Women Quarterly}}.]
The \emph{Psychology of Women Quarterly (PWQ)} is the official journal
of Division 35. PWQ is a feminist, scientific, peer-reviewed journal
that publishes empirical research, critical reviews and theoretical
articles that advance a field of inquiry, brief reports on timely
topics, teaching briefs, and invited book reviews related to the
psychology of women and gender. Topics include (but are not limited to)
feminist approaches, methodologies, and critiques; violence against
women; body image and objectification; sexism, stereotyping, and
discrimination; intersectionality of gender with other social locations
(such as age, ability status, class, ethnicity, race, and sexual
orientation); international concerns; lifespan development and change;
physical and mental well being; therapeutic interventions; sexuality;
social activism; and career development.
\item[\href{http://psychosocial-intervention.elsevier.es/en/vol-23-num-01/sumario/13018163/\#.U1p7nOZdXTM}{\textbackslash{}textsc\{Psychosocial
Intervention}.]
The Psychosocial Intervention is a peer-reviewed journal that publishes
papers in all areas relevant to psychosocial intervention at the
individual, family, social networks, organization, community, and
population levels. The Journal emphasizes an evidence-based perspective
and welcomes papers reporting original basic and applied research,
program evaluation, and intervention results. The journal will also
feature integrative reviews, and specialized papers on theoretical
advances and methodological issues.
\item[\href{http://www.journals.elsevier.com/social-science-and-medicine/}{\textsc{Social Science \& Medicine}}.]
The \emph{Social Science \& Medicine} Journal provides an international
and interdisciplinary forum for the dissemination of social science
research on health. We publish original research articles (both
empirical and theoretical), reviews, position papers and commentaries on
health issues, to inform current research, policy and practice in all
areas of common interest to social scientists, health practitioners, and
policy makers. The journal publishes material relevant to any aspect of
health from a wide range of social science disciplines (anthropology,
economics, epidemiology, geography, policy, psychology, and sociology),
and material relevant to the social sciences from any of the professions
concerned with physical and mental health, health care, clinical
practice, and health policy and organization. We encourage material
which is of general interest to an international readership.
\item[\href{http://www.scra27.org/publications/tcp/}{\textsc{The Community Psychologist}}.]
\emph{The Community Psychologist (TCP)} is the Society for Community
Research and Action's newsletter, provides an informal venue for SCRA
members to exchange ideas and information. It includes briefreports and
reflections on the practice of community research and action, as well as
updates and announcements of interest to members. Regular features
include columns on women's issues, public policy, international issues,
students' concerns, racial and cultural affairs, and job opportunities.
SCRA members receive three to four issues of TCP each year.
\item[\href{http://tps.sagepub.com/}{\textsc{Transcultural Psychiatry}}.]
The \emph{Transcultural Psychiatry} is a fully peer reviewed
international journal that publishes original research and review
articles on cultural psychiatry and mental health. Cultural psychiatry
is concerned with the social and cultural determinants of
psychopathology and psychosocial treatments of the range of mental and
behavioural problems in individuals, families and communities. In
addition to the clinical research methods of psychiatry, it draws from
the disciplines of psychiatric epidemiology, medical anthropology and
cultural psychology.
\item[\href{https://www.press.jhu.edu/journals/progress_in_community_health_partnerships/}{\textsc{Progress in Community Health Partnerships: Research, Education, \& Action}}.]
\emph{Progress in Community Health Partnerships (PCHP)} is a national,
peer-reviewed journal whose mission is to identify and publicize model
programs that use community partnerships to improve public health,
promote progress in the methods of research and education involving
community health partnerships, and stimulate action that will improve
the health of people and communities. The first scholarly journal
dedicated to Community-Based Participatory Research (CBPR), PCHP is a
must for public health professionals and the libraries that serve them
\end{description}

\newpage



\end{document}
